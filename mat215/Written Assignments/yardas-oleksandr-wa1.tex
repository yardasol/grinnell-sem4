\documentclass[12pt]{article}
\usepackage{latexsym, amssymb, amsmath, amsfonts, amscd, amsthm}
\usepackage[margin=1in]{geometry}
\linespread{1} %Change the line spacing only if instructed to do so.

\newenvironment{problem}[2][Problem]
{
	\begin{trivlist} 
		\item[\hskip \labelsep {\bfseries #1 #2:}]
	}
{
	\end{trivlist}
	}

\newenvironment{solution}[1][Solution]
{
	\begin{trivlist} 
		\item[\hskip \labelsep {\itshape #1:}]
	}
	{
	\end{trivlist}
}

\newenvironment{collaborators}[1][Collaborator(s)]
{
	\begin{trivlist} 
		\item[\hskip \labelsep {\bfseries #1:}]
	}
	{
	\end{trivlist}
}

%%%%%%%%%%%%%%%%%%%%%%%%%%%%%%%%%%%%%%%%%%%%%%%%%%
%%%%%%%%%%%%%%%%%%%%%%%%%%%%%%%%%%%%%%%%%%%%%%%%%%
%%%%%%%%%%%%%%%%%%%%%%%%%%%%%%%%%%%%%%%%%%%%%%%%%%
%
%
%    You need only modify code below this block.
%
%
%%%%%%%%%%%%%%%%%%%%%%%%%%%%%%%%%%%%%%%%%%%%%%%%%%
%%%%%%%%%%%%%%%%%%%%%%%%%%%%%%%%%%%%%%%%%%%%%%%%%%
%%%%%%%%%%%%%%%%%%%%%%%%%%%%%%%%%%%%%%%%%%%%%%%%%%
%
%
%%%%%%%%%%%%%%%
%
% Modify the title, author, and due date for each assignment:
%
%%%%%%%%%%%%%%%
\title{MAT215\\Written Assignment 1}
\author{Oleksandr Yardas}
\date{Due: 02/02/2018}
%%%%%%%%%%%%%%%
%
% Do not modify:
%
%%%%%%%%%%%%%%%
\begin{document}
\maketitle
\newpage
%%%%%%%%%%%%%%%
%
% Your problem statements and solutions start here.
% Use the \newpage command between problems so that
% each of your problems begins on its own page.
%
%%%%%%%%%%%%%%%
%Provide the problem statement.
\noindent
{\bf Definition:}  Let $a \in \mathbb{Z}$.
\begin{itemize}
	\item We say that $a$ has {\em type 0} if there exists $m \in \mathbb{Z}$ with $a = 4m$.
	\item We say that $a$ has {\em type 1} if there exists $m \in \mathbb{Z}$ with $a = 4m+1$.
	\item We say that $a$ has {\em type 2} if there exists $m \in \mathbb{Z}$ with $a = 4m+2$.
	\item We say that $a$ has {\em type 3} if there exists $m \in \mathbb{Z}$ with $a = 4m+3$.
\end{itemize}
\noindent
Just like for the evens, it is possible to show that every integer is of either type 0, or type 1, or type 2, or type 3.  Feel free to use this result in your arguments.

\vspace{10pt}

\begin{problem}{1}
	Show that for all $a \in \mathbb{Z}$, we have that $a^2$ either has type 0 or has type 1.
\end{problem}
\begin{solution}
	Let $a \in \mathbb{Z}$ be arbitrary. By Fact 1.4.6, $a$ is either even or odd. 
	Consider the case in which $a$ is odd. We then have:

	\begin{align*}
	a &= 2m + 1                 &  \it{(By \: defintion)}\\
	a^2 &= (2m + 1)^2\\
	&= 4m^2 + 4m +1\\
	&= 4(m^2 + m) + 1
	\end{align*}
	
	Because $a \in \mathbb{Z}$, $a^2 \in \mathbb{Z}$. Likewise, $m \in \mathbb{Z}$, so $m^2 \in \mathbb{Z}$ and $m^2 + m \in \mathbb{Z}$. Thus, $a^2 = 4(m^2+m) + 1$ satisfies the definition of {\em type 1}.
	
	Consider now the case in whihc $a$ is even. We then have:
	\begin{align*}
	a &= 2m                 &  \it{(By \: defintion)}\\
	a^2 &= (2m)^2\\
	&= 4m^2\\
	&= 4(m^2)
	\end{align*}

	Because $a \in \mathbb{Z}$, $a^2 \in \mathbb{Z}$. Likewise, $m \in \mathbb{Z}$, so $m^2 \in \mathbb{Z}$. Thus, $a^2 = 4(m^2)$ satisfies the definition of {\em type 0}. Becuase $a$ was arbitrary, the result follows.
	
\end{solution}
\begin{collaborators}
	Stephen Cropper (help with formatting)
\end{collaborators}

\newpage

\begin{problem}{2}
	In this problem, you will prove the following:
	\[
	``\text{If } a \in \mathbb{Z} \text{ and } a \text{ has type } 0, \text{ then there exists } b,c \in \mathbb{Z} \text{ with } a = b^2 - c^2".
	\]
	However, we will do it in stages.
	
	\noindent
	a. Write down some examples of type $0$ integers.  For each of these, find examples of $b$ and $c$ with $a = b^2 - c^2$.
	\begin{solution}
	The integers which immediately come to mind are $4, 8 ,12$ and $16$.
	\begin{align*}
	4=4*1\\
	8=4*2\\
	12=4*3\\
	16=4*4
	\end{align*}
	
	We can see that each of the selected integers is of {\em type 1}. We can also see that:
	
	\begin{align*}
	4=2^2 - 0^2\\
	8=3^2 - 1^2 \\
	12=4^2 - 2^2\\
	16=5^2 - 3^2
	\end{align*}

	Which agree with the initial statement for the integers 4, 8, 12, and 16.
	
	\end{solution}
	
	\noindent
	b. Looking at your examples, make a guess as to a general pattern.  In other words, if we have a type $0$ integer $a$ and we fix $n \in \mathbb{Z}$ with $a = 4n$, what do you guess will work for $b$ and $c$? $\{x : x\}$
	\begin{solution}
	In Part a), we found that:
		\begin{align*}
	4=2^2 - 0^2\\
	8=3^2 - 1^2 \\
	12=4^2 - 2^2\\
	16=5^2 - 3^2
	\end{align*}
	
	The pattern seems to be $4n=(n+1)^2 - (n-1)^2$.
	\end{solution}

	\noindent
	c. Now write up a formal proof of the statement. \\
\end{problem}

\begin{solution}
	Let $a \in \mathbb{Z}$ be arbitrary and be of {\em type 1}. So we can fix an $n \in \mathbb{Z}$ with $a=4n$. So we have:
	\begin{align*}
	a&= 4n\\
	 &=2n + 2n\\
	 &=2n+1 + 2n -1\\
	 &=n^2 + 2n + 1 - n^2 + 2n -1\\
	 &=(n+1)^2 - (n-1)^2
	\end{align*}
	Because $n \in \mathbb{Z}$, we have proven the existence of a $b,c \in \mathbb{Z}$ with $a=b^2-c^2$. Because $a$ was arbitrary, the result follows.


\end{solution}
\begin{collaborators}
	Jill Rix (Formatting)
\end{collaborators}
\end{document}