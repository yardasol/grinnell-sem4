\documentclass[12pt]{article}
\usepackage{latexsym, amssymb, amsmath, amsfonts, amscd, amsthm}
\usepackage{framed}
\usepackage[margin=1in]{geometry}
\linespread{1} %Change the line spacing only if instructed to do so.

\newenvironment{problem}[2][Problem]
{
	\begin{trivlist} 
		\item[\hskip \labelsep {\bfseries #1 #2:}]
	}
{
	\end{trivlist}
	}

\newenvironment{solution}[1][Solution]
{
	\begin{trivlist} 
		\item[\hskip \labelsep {\itshape #1:}]
	}
	{
	\end{trivlist}
}

\newenvironment{collaborators}[1][Collaborator(s)]
{
	\begin{trivlist} 
		\item[\hskip \labelsep {\bfseries #1:}]
	}
	{
	\end{trivlist}
}

%%%%%%%%%%%%%%%%%%%%%%%%%%%%%%%%%%%%%%%%%%%%%%%%%%
%%%%%%%%%%%%%%%%%%%%%%%%%%%%%%%%%%%%%%%%%%%%%%%%%%
%%%%%%%%%%%%%%%%%%%%%%%%%%%%%%%%%%%%%%%%%%%%%%%%%%
%
%
%    You need only modify code below this block.
%
%
%%%%%%%%%%%%%%%%%%%%%%%%%%%%%%%%%%%%%%%%%%%%%%%%%%
%%%%%%%%%%%%%%%%%%%%%%%%%%%%%%%%%%%%%%%%%%%%%%%%%%
%%%%%%%%%%%%%%%%%%%%%%%%%%%%%%%%%%%%%%%%%%%%%%%%%%
%
\title{Assignment: Writing Assignment 2} %Change this to the assignment you are submitting.
\author{Name: Oleksandr Yardas} %Change this to your name.
\date{Due Date: 02/09/2018 } %Change this to the due date for the assignment you are submitting.
\begin{document}
	\maketitle
	\thispagestyle{empty}
	
	\section*{List Your Collaborators:}%Enter your collaborators names below. Do not delete extra rows.
	
	\begin{itemize}
		\begin{framed}
			\item 
			Problem 1: None
			\\\\
		\end{framed}
		\begin{framed}
			\item 
			Problem 2: None
			\\\\
		\end{framed}
		\begin{framed}
			\item 
			Problem 3: None 
			\\\\
		\end{framed}
		\begin{framed}
			\item 
			Problem 4: Not Applicable
			\\\\
		\end{framed}
		\begin{framed}
			\item 
			Problem 5: Not Applicable
			\\\\
		\end{framed}
		\begin{framed}
			\item 
			Problem 6: Not Applicable
			\\\\
		\end{framed}
	\end{itemize}
\newpage
%
%%%%%%%%%%%%%%%
%
% Your problem statements and solutions start here.
% Use the \newpage command between problems so that
% each of your problems begins on its own page.
%
%%%%%%%%%%%%%%%
%Provide the problem statement.
\begin{problem}{1}
	Consider the function $f : \mathbb{Z} \to \mathbb{Z}^2$ given by $f(n) = (3n^2 -77, 5n + 6)$.
	\newline
	\newline
	\noindent a. Is $f$ injective? Justify your answer carefully.
\begin{solution}
	We assume that $f$ is injective. Let $a,b \in \mathbb{Z}$ be arbitrary. By definition of injective, whenever $a, b \in \mathbb{Z}$ satisfy $f(a)=f(b)$, we have that $a=b$. So we get
	\[
	(3a^2-77,5a+6) = (3b^2-77,5b+6)
	\]
	We will look at each element of the ordered pair separately. For the second element of the ordered pair, we have:
	\begin{align*}
	5a+6=5b+6\\
	5a=5b\\
	a=b
	\end{align*}
	We must now show that this is also the case for the first element of the ordered pair in order for $f$ to satisfy the definition of injective.
	For the first element of the ordered pair, we have:
	\begin{align*}
	3a^2-77=3b^2-77\\
	3a^2=3b^2\\
	a^2=b^2
	\end{align*}
	By definition, for all $z \in \mathbb{Z}$, $z^2 \geq 0$. So for our arbitrary $a,b \in \mathbb{Z}$ with $a^2 = b^2$, we conclude there are two solutions: $a=b$, $a=-b$. But we assumed that $f$ was injective, and this assumption has led to a conclusion that contradicts the definition of injective. Because $a,b$ were arbitrary, it must be the case that $f$ is not injective.
	 \end{solution}

\noindent b. Is $f$ surjective? Justify your answer carefully.
\begin{solution}
	We assume that $f$ is surjective. By the definition of surjective, for all $(x,y) \in \mathbb{Z}^2$, there exists a $n \in \mathbb{Z}$ such that $f(n)=(x,y)$. Let $(x,y)=(-80, 0)$. So we get:
	\[
	3n^2-77= -80 \text{ and } 5n+6=-6
	\]
	Solving for $n$, we find that:
	\begin{align*}
	3n^2=-3 \text{      and     } 5n=-6\\
	n^2=-1 \text{      and      } n=-\frac{6}{5}\\
	n=\pm \sqrt{}{-1} \text{      and      } n=-\frac{6}{5}
	\end{align*}
	We conclude that for $(x,y)=(-80,0)$, we have $n=+\sqrt{}{-1}$, $n=-\sqrt{}{-1}$ for the first element and $n=-\frac{6}{5}$ for the second element. Note that for all of these values, $n \notin \mathbb{Z}$. Also note that we have two values of $n$ for the first element, and completely different value of $n$ for the second element. But we assumed that $f$ was surjective, that is, that for all $(x,y) \in \mathbb{Z}^2$, there exists {\bf a} $n \in \mathbb{Z}$ such that $f(n)=(x,y)$, however we have found an $(x,y) \in \mathbb{Z}^2$, $(-80,0)$, such that for all $n \in \mathbb{Z}$, $f(n) \neq (-80,0)$, so it must be the case that $f$ is not surjective.	
	
\end{solution}
\end{problem}

\newpage
\begin{problem}{2}
	Suppose that $A, B,$ and $C$ are sets and that both $f:A \to B$ and $g:B \to C$ are surjective functions. Show that the function $g \circ f : A \to C$ is surjective.
	
	\noindent {\it Hint:} You are trying to prove that the function $g \circ f : A \to C$ is surjective. Following the guidelines before the proof of Proposition 1.6.9, you should start by taking an arbitrary $c \in C$. With this $c$ in hand, you goal is to build an $a \in A$ with $(g \circ f)(a) = c$.
\begin{solution} %{\bf Definition } {\bf 1.6.3. } For a function function $h : X \to Y$, we define {\it range}$(h)=\{y \in Y : \text{ There exists } x \in X \text{ with } f(x)=y\}$.

%We first restate the definition of surjective. We say a function function $h : X \to Y$ is surjective exactly when {\it range}$(h)=Y$.  So, by this restatement of the definition of surjective, {\it range}$(f)=B$ and {\it range}$(g)=C$. For all $b \in B$, there exists an $a \in A$ such that $f(a)=b$. For all $c \in C$, there exists a $b \in B$ with  
Let $c \in C$ be arbitrary.
%Let $c \in A$ %, $b \in B$, $c \in C$ be arbitrary. 
%For all $b \in B$, we can fix an $a \in A$ such that $f(a)=b$. 
$g$ is surjective, so for any $c \in C$, there exists a $b \in B$ such that $g(b)=c$. $f$ is surjective, so for every $b \in B$, there exists an $a \in A$ such that $f(a)=b$. Notice that for each $b \in B$ with $g(b)=c$, there is an $a \in A$ with $f(a)=b$. So we can say that for every $c \in C$, there exists an $a \in A$ such that $g(f(a))=c$. This function satisfies the definition of surjective. By the definition of function composition, $(g \circ f)(a)=g(f(a))$ for all $a \in A$. So $g(f(a))=(g \circ f)(a)$ is surjective.
\end{solution}

\end{problem}

\newpage
\begin{problem}{3}
	Suppose that we have a function $f : \mathbb{R} \to \mathbb{R}$ with the property that $f(x \cdot y) = f(x) \cdot f(y)$ for all $x,y \in \mathbb{R}$. Suppose that $f(2)=5$ and $f(3) = 7$. What is $f(\frac{1}{6})$? Explain.
	
\noindent {\it Hint:} What can you say about $f(1)$?

\begin{solution}
By the definition of $f$, we have
\begin{align*}
f(2)=f(4 \cdot \frac{1}{2})=f(4)\cdot f(\frac{1}{2})\\
\frac{f(2)}{f(4)}=f(\frac{1}{2})\\
\frac{f(2)}{f(2) \cdot f(2)}=f(\frac{1}{2})\\
\frac{1}{f(2)}=f(\frac{1}{2})
\end{align*}
and 
\begin{align*}
f(3)=f(9 \cdot \frac{1}{3})=f(9)\cdot f(\frac{1}{3})\\
\frac{f(3)}{f(9)}=f(\frac{1}{3})\\
\frac{f(3)}{f(3) \cdot f(3)}=f(\frac{1}{3})\\
\frac{1}{f(3)}=f(\frac{1}{3})
\end{align*}
Notice that
\begin{align*}
f(\frac{1}{6})=f(\frac{1}{2} \cdot \frac{1}{3})=f(\frac{1}{2})\cdot f(\frac{1}{3})\\
=\frac{1}{f(2)} \cdot \frac{1}{f(3)}\\
=\frac{1}{5} \cdot \frac{1}{7}\\
=\frac{1}{35}
\end{align*}
So $f(\frac{1}{6})=\frac{1}{35}$.
\end{solution}
\end{problem}

\end{document}