\documentclass[12pt]{article}
\usepackage{latexsym, amssymb, amsmath, amsfonts, amscd, amsthm, xcolor, pgfplots}
\usepackage{framed}
\usepackage[margin=1in]{geometry}
\linespread{1} %Change the line spacing only if instructed to do so.

\newenvironment{problem}[2][Problem]
{
	\begin{trivlist} 
		\item[\hskip \labelsep {\bfseries #1 #2:}]
	}
{
	\end{trivlist}
	}

\newenvironment{solution}[1][Solution]
{
	\begin{trivlist} 
		\item[\hskip \labelsep {\itshape #1:}]
	}
	{
	\end{trivlist}
}

\newenvironment{collaborators}[1][Collaborator(s)]
{
	\begin{trivlist} 
		\item[\hskip \labelsep {\bfseries #1:}]
	}
	{
	\end{trivlist}
}

%%%%%%%%%%%%%%%%%%%%%%%%%%%%%%%%%%%%%%%%%%%%%%%%%%
%%%%%%%%%%%%%%%%%%%%%%%%%%%%%%%%%%%%%%%%%%%%%%%%%%
%%%%%%%%%%%%%%%%%%%%%%%%%%%%%%%%%%%%%%%%%%%%%%%%%%
%
%
%    You need only modify code below this block.
%
%
%%%%%%%%%%%%%%%%%%%%%%%%%%%%%%%%%%%%%%%%%%%%%%%%%%
%%%%%%%%%%%%%%%%%%%%%%%%%%%%%%%%%%%%%%%%%%%%%%%%%%
%%%%%%%%%%%%%%%%%%%%%%%%%%%%%%%%%%%%%%%%%%%%%%%%%%
%
\title{Assignment: Written Assignment 4} %Change this to the assignment you are submitting.
\author{Name: Oleksandr Yardas} %Change this to your name.
\date{Due Date: 03/02/2018 } %Change this to the due date for the assignment you are submitting.
\begin{document}
	\maketitle
	\thispagestyle{empty}
	
	\section*{List Your Collaborators:}%Enter your collaborators names below. Do not delete extra rows.
	
	\begin{itemize}
		\begin{framed}
			\item 
			Problem 1: None
			\\\\
		\end{framed}
		\begin{framed}
			\item 
			Problem 2: None
			\\\\
		\end{framed}
		\begin{framed}
			\item 
			Problem 3: None
			\\\\
		\end{framed}
		\begin{framed}
			\item 
			Problem 4: Not Applicable
			\\\\
		\end{framed}
		\begin{framed}
			\item 
			Problem 5: Not Applicable
			\\\\
		\end{framed}
		\begin{framed}
			\item 
			Problem 6: Not Applicable
			\\\\
		\end{framed}
	\end{itemize}
\newpage
%
%%%%%%%%%%%%%%%
%
% Your problem statements and solutions start here.
% Use the \newpage command between problems so that
% each of your problems begins on its own page.
%
%%%%%%%%%%%%%%%

%FORMATTING OPTIONS
%FOR BLANK SPACES: \underline{\hspace{2cm}}
%FOR SPACES IN align OR SIMILAR ENVIRONMENTS:  \hphantom{1000}
%FOR MATRICES: \begin{matrix} \end{matrix}, can add p, b, B, v, V, small as suffix to "matrix"
%SETS: \mathbb{R}^, :\mathbb{R}^ \to \mathbb{R}^
%Vectors: \vec{},
%SUBSCRIPTS: _{}
%FRACTIONS: \frac{}{}

%Provide the problem statement.

\begin{problem}{1}
Suppose that $T:\mathbb{R}^2 \to \mathbb{R}^2$ is a surjective linear transformation and that $\vec{u_{1}},\vec{u_{2}} \in \mathbb{R}^2$. Show that if Span$(\vec{u_{1}},\vec{u_{2}}) = \mathbb{R}^2$, then Span$(T(\vec{u_{1}}),T(\vec{u_{2}}))=\mathbb{R}^2$.
\noindent
\newline
\newline
{\it Hint:} You need only show that $\mathbb{R}^2 \subseteq$ Span$(T(\vec{u_{1}}),T(\vec{u_{2}}))$, as the reverse containment is immediate. Start by taking an arbitrary $\vec{w} \in \mathbb{R}^2$. To show that $\vec{w} \in$ Span$(T(\vec{u_{1}}),T(\vec{u_{2}})$, what do you need to do?
\noindent
\newline
\newline

\begin{solution}
We have that Span$(\vec{u_{1}},\vec{u_{2}}) =\mathbb{R}^2$. Applying Theorem 2.3.10, it follows that that there does not exist $n \in \mathbb{R}$ with $n\vec{u_{1}}=\vec{u_{2}}$. %that is, $\vec{u_{1}}$ and $\vec{u_{2}}$ are linearly independent. 
Because $T$ is surjective, it follows from Corollary 3.3.5 that $T$ is bijective, that is, that there is exactly one $\vec{v} \in \mathbb{R}^2$ for every $\vec{u} \in \mathbb{R}^2$ with $\vec{u}=T(\vec{v})$. Combining this with the fact that that there does not exist $n \in \mathbb{R}$ with $n\vec{u_{1}}=\vec{u_{2}}$, it must be the case that that there does not exist $m \in \mathbb{R}$ with $mT(\vec{u_{1}})=T(\vec{u_{2}})$ (if we assume that it were the case, this would imply $m\vec{u_{1}}=\vec{u_{2}}$ because $T$ is bijective, but we know that there does not exist $n \in \mathbb{R}$ with $n\vec{u_{1}}=\vec{u_{2}}$, so we have a contradiction and so the previous statement about $T$ follows). Applying Theorem 2.3.10, it follows that Span$(T(\vec{u_{1}}),T(\vec{u_{2}})) = \mathbb{R}^2$. 

%If this is not a sufficient conclusion, note that is immediate that Span$(T(\vec{u_{1}}),T(\vec{u_{2}})) \subseteq \mathbb{R}^2$, so we just need to show that $\mathbb{R}^2 \subseteq$ Span$(T(\vec{u_{1}}),T(\vec{u_{2}}))$. Let $\vec{w} \in \mathbb{R}^2$ be arbitrary. Let $\vec{v_{1}}, \vec{v_{2}} \in$ Span$(T(\vec{u_{1}}),T(\vec{u_{2}}))$ be arbitrary. By the definition of Span, we can fix $x,y,p,q \in \mathbb{R}$ such that $\vec{v_{1}}=x\cdot T(\vec{u_{1}}) + y \cdot T(\vec{u_{2}})$, and $\vec{v_{2}}=p\cdot T(\vec{u_{1}}) + q \cdot T(\vec{u_{2}})$. Span$(T(\vec{u_{1}}),T(\vec{u_{2}}))\subseteq \mathbb{R}^2$, so $\vec{v_{1}},\vec{v_{2}} \in \mathbb{R}^2$. Because $\vec{w} \in \mathbb{R}^2$, we can fix an $a,b \in \mathbb{R}^2$ with $\vec{w} = a \vec{v_{1}} + b\vec{v_{2}} =a\cdot (x\cdot T(\vec{u_{1}}) + y \cdot T(\vec{u_{2}})) + b\cdot (p\cdot T(\vec{u_{1}}) + q \cdot T(\vec{u_{2}})) = (ax+bp) \cdot T(\vec{u_{1}}) + (ay+bq) \cdot T(\vec{u_{2}})$. Because $ax+bp,ay+bq \in \mathbb{R}$, $\vec{w} \in$ Span$(T(\vec{u_{1}}),T(\vec{u_{2}}))$. Because $\vec{w}$ was arbitrary, the result follows.
\end{solution}
%\newline
%\newline
%\newline
%\newline
%\newline
%\newline
%\[
%\text{PAGE 1 OF X FOR PROBLEM 1}
%\]
\end{problem}






\newpage
\begin{problem}{2}
Suppose that $T:\mathbb{R}^2 \to \mathbb{R}^2$ is a surjective linear transformation and that $\vec{u},\vec{w} \in \mathbb{R}^2$. Show that if $\vec{w}\notin$ Span$(\vec{u})$, then $T(\vec{w}) \notin$ Span$(T(\vec{u}))$.
\noindent
\newline
\newline

\begin{solution}
Let $\vec{u}, \vec{w} \in \mathbb{R}^2$ be arbitrary with the property that $\vec{w} \notin$ Span$(\vec{u})$. Because $\vec{w} \notin$ Span$(\vec{u})$, by definition there does not exist a $c\in \mathbb{R}$ with $c \cdot \vec{u} = \vec{w}$. We assume that there exists a surjective linear transformation $T:\mathbb{R}^2 \to \mathbb{R}^2$ with $T(\vec{w}) \in$ Span$(T(\vec{u}))$. Because $T(\vec{w}) \in$ Span$(T(\vec{u}))$, we can fix a $d \in \mathbb{R}$ such that $T(\vec{w}) = d\cdot T(\vec{u})$ (by definition of Span). We conclude that $T(\vec{w}) = T(d\cdot \vec{u})$ (by the definition of linear transformation). Because $T$ is surjective, it follows from Corollary 3.3.5 that $T$ is bijective, that is, that for there is exactly one $\vec{v} \in \mathbb{R}^2$ for every $\vec{u} \in \mathbb{R}^2$ with $\vec{u}=T(\vec{v})$. So it must be the case that $\vec{w} = d\cdot \vec{u}$. But we have defined $\vec{u}, \vec{w}$ such that that there does not exist a $c\in \mathbb{R}$ with $c \cdot \vec{u} = \vec{w}$. Our assumption that there exists a surjective linear transformation $T:\mathbb{R}^2 \to \mathbb{R}^2$ with $T(\vec{w}) \in$ Span$(T(\vec{u}))$ has led us to a contradiction, so it must be the case that $T(\vec{w}) \notin$ Span$(T(\vec{u}))$.
\end{solution}
%\newline
%\newline
%\newline
%\newline
%\newline
%\newline
%\[
%\text{PAGE 1 OF X FOR PROBLEM 2}
%\]
\end{problem}






\newpage
\begin{problem}{3}
In this problem, we determine which 2$\times$2 matrices commute with {\it every} 2$\times$2 matrix.
\noindent
\newline
a. Show that if $r \in \mathbb{R}$ and we let
\[
A=\begin{pmatrix}r&0\\0&r\end{pmatrix}\text{,}
\]
then $AB=BA$ for every 2$\times$2 matrix $B$.
\begin{solution}
Let B be an arbitrary 2$\times$2 matrix, and fix $a,b,c,d \in \mathbb{R}$ with $B= \begin{pmatrix} a & b \\ c& d \end{pmatrix}$. Note that:
\begin{align*}
AB =& \begin{pmatrix}r&0\\0&r\end{pmatrix} \cdot \begin{pmatrix} a & b \\ c& d \end{pmatrix} & \text{(By definition of $A,B$)}\\
=& \begin{pmatrix} ra + 0\cdot c & rb + 0\cdot d \\ 0\cdot a + rc & 0\cdot b + rd\end{pmatrix} & \text{(By definition 3.2.1)}\\
=&\begin{pmatrix} ra +  0 & rb + 0 \\ 0 + rc &  0 + rd \end{pmatrix} =\begin{pmatrix} ar +  0 & br + 0 \\ 0 + cr &  0 + dr \end{pmatrix} & \\
=&\begin{pmatrix} ar +  0 & 0 + br \\ cr + 0 &  0 + dr \end{pmatrix} =\begin{pmatrix} ar + b\cdot 0 & a\cdot 0 + br \\ cr + d\cdot 0 &  c\cdot 0 + dr \end{pmatrix} & \\
=& \begin{pmatrix} a & b \\ c& d \end{pmatrix} \cdot \begin{pmatrix}r&0\\0&r\end{pmatrix} & \text{(By definition 3.2.1)}\\
=& BA
\end{align*}
Therefore, $AB=BA$. Because $B$ was an arbitrary 2$\times$2 matrix, the result follows.
\end{solution}
\noindent
\newline
\newline
b. Suppose that $A$ is a 2$\times$2 matrix with the property that $AB=BA$ for every 2$\times$2 matrix $B$. Show that there exists $r \in \mathbb{R}$ such that
\[
A=\begin{pmatrix}r&0\\0&r\end{pmatrix}\text{.}
\]
\begin{solution}
Let $A,B$ be arbitrary 2$\times$2 matrices, and fix $a_{1,1},a_{1,2},a_{2,1},a_{2,2},b_{1,1},b_{1,2},b_{2,1},b_{2,2} \in \mathbb{R}$ such that $A=\begin{pmatrix} a_{1,1} & a_{1,2} \\ a_{2,1} & a_{2,2} \end{pmatrix},B=\begin{pmatrix} b_{1,1} & b_{1,2} \\ b_{2,1} & b_{2,2} \end{pmatrix}$. We assume that that $AB=BA$, so we get:
\begin{align*}
AB=& BA &\\
\Rightarrow \begin{pmatrix} a_{1,1} & a_{1,2} \\ a_{2,1} & a_{2,2} \end{pmatrix} \cdot \begin{pmatrix} b_{1,1} & b_{1,2} \\ b_{2,1} & b_{2,2} \end{pmatrix} =&\begin{pmatrix} b_{1,1} & b_{1,2} \\ b_{2,1} & b_{2,2} \end{pmatrix} \cdot \begin{pmatrix} a_{1,1} & a_{1,2} \\ a_{2,1} & a_{2,2} \end{pmatrix}  & \text{(By definition of $A,B$)} \\
\Rightarrow \begin{pmatrix} a_{1,1}b_{1,1} + a_{1,2}b_{2,1} & a_{1,1}b_{1,2} + a_{1,2}b_{2,2} \\ a_{2,1}b_{1,1} + a_{2,2}b_{2,1} & a_{2,1}b_{1,2} + a_{2,2}b_{2,2} \end{pmatrix} =& \begin{pmatrix} b_{1,1}a_{1,1} + b_{1,2}a_{2,1} & b_{1,1}a_{1,2} + b_{1,2}a_{2,2} \\ b_{2,1}a_{1,1} + b_{2,2}a_{2,1} & b_{2,1}a_{1,2} + b_{2,2}a_{2,2} \end{pmatrix} &  \text{(By definition 3.2.1)}\\
\end{align*}
\[
\text{PAGE 1 OF 2 FOR PROBLEM 3}
\]
Let $a_{1,1}=a_{2,2}=r, a_{1,2}=a_{2,1}=0$, where $r\in \mathbb{R}$ is some fixed constant. So $A=\begin{pmatrix}r&0\\0&r\end{pmatrix}$. We test to see if $AB=BA$ is true
\begin{align*}
\begin{pmatrix} rb_{1,1} + 0b_{2,1} & rb_{1,2} + 0b_{2,2} \\ 0b_{1,1} + rb_{2,1} & 0b_{1,2} + rb_{2,2} \end{pmatrix} =& \begin{pmatrix} b_{1,1}r + b_{1,2}0 & b_{1,1}0 + b_{1,2}r \\ b_{2,1}r + b_{2,2}0 & b_{2,1}0 + b_{2,2}r \end{pmatrix}\\
\Rightarrow \begin{pmatrix} rb_{1,1} & rb_{1,2} \\ rb_{2,1} & rb_{2,2} \end{pmatrix} =& \begin{pmatrix} b_{1,1}r  &  b_{1,2}r \\ b_{2,1}r & b_{2,2}r \end{pmatrix} \text{,}\\
\end{align*}
and it is!. We have found an $r\in \mathbb{R}$ such that $A=\begin{pmatrix}r&0\\0&r\end{pmatrix}$ and has the property that $AB=BA$ for every  2$\times$2 matrix $B$.
%It follows that :
%\begin{align*}
%& a_{1,1}b_{1,1} + a_{1,2}b_{2,1} =b_{1,1}a_{1,1} + b_{1,2}a_{2,1} &\\
%&\Rightarrow a_{1,2}b_{2,1} = b_{1,2}a_{2,1} & \text{(1)}\\
%\end{align*}
%\begin{align*}
%& a_{1,1}b_{1,2} + a_{1,2}b_{2,2} = b_{1,1}a_{1,2} + b_{1,2}a_{2,2} &\\
%&\Rightarrow (a_{1,1}-a_{2,2})b_{1,2} = (b_{1,1} -b_{2,2})a_{1,2} &\text{(2)}\\
%\end{align*}
%\begin{align*}
%& a_{2,1}b_{1,1} + a_{2,2}b_{2,1}=b_{2,1}a_{1,1} + b_{2,2}a_{2,1} &\\ 
%&\Rightarrow a_{2,1}(b_{1,1} -b_{2,2})=b_{2,1}(a_{1,1}-a_{2,2})&\text{(3)}\\
%\end{align*}
%\begin{align*}
%&a_{2,1}b_{1,2} + a_{2,2}b_{2,2} = b_{2,1}a_{1,2} + b_{2,2}a_{2,2} & \\
%&\Rightarrow a_{2,1}b_{1,2}  = b_{2,1}a_{1,2} & \text{(4)}
%\end{align*}
%(1) and (4) are identical. If we assume (1) is true, note we can reduce (2) and (3) to:
%\begin{align*}
%&b_{1,1} -b_{2,2} = a_{2,1} & \text{from (2)}\\
%&a_{1,1}-a_{2,2}= b_{2,1} & \text{from (2)} \\
%&b_{1,1} -b_{2,2} =a_{1,2}& \text{from (3)}\\
%&a_{1,1}-a_{2,2}=b_{1,2} & \text{from (3)}
%\end{align*}
%So it must be the case that $ a_{2,1} =  a_{1,2}$ and that $b_{2,1}=b_{1,2}$.
%\begin{align*}
%a_{1,2}b_{2,1} = b_{1,2}a_{2,1}\\
%(a_{1,1}-a_{2,2})b_{1,2} = (b_{1,1} -b_{2,2})a_{1,2}\\
%a_{2,1}(b_{1,1} -b_{2,2})=b_{2,1}(a_{1,1}-a_{2,2})\\
%a_{2,1}b_{1,2}  = b_{2,1}a_{1,2}
%\end{align*}
%So we have:
%\[
%A= \begin{pmatrix} a_{1,1} & a_{1,2}\\ \frac{a_{1,2}b_{2,1}}{b_{1,2}} & a_{1,1}-\frac{a_{1,2}(b_{1,1}-b_{2,2})}{b_{1,2}} \end{pmatrix}
%\]
\newline
\newline
\newline
\newline
\newline
\newline
\newline
\newline
\newline
\newline
\newline
\newline
\newline
\newline
\newline
\newline
\newline
\newline
\newline
\newline
\newline
\newline
\newline
\newline
\newline
\newline
\newline
\newline
\newline
\newline
\newline
\newline
\[
\text{PAGE 2 OF 2 FOR PROBLEM 3}
\]
\end{solution}
\end{problem}


\end{document}