\documentclass[12pt]{article}
\usepackage{latexsym, amssymb, amsmath, amsfonts, amscd, amsthm}
\usepackage{framed}
\usepackage[margin=1in]{geometry}
\linespread{1} %Change the line spacing only if instructed to do so.

\newenvironment{problem}[2][Problem]
{
	\begin{trivlist} 
		\item[\hskip \labelsep {\bfseries #1 #2:}]
	}
{
	\end{trivlist}
	}

\newenvironment{solution}[1][Solution]
{
	\begin{trivlist} 
		\item[\hskip \labelsep {\itshape #1:}]
	}
	{
	\end{trivlist}
}

\newenvironment{collaborators}[1][Collaborator(s)]
{
	\begin{trivlist} 
		\item[\hskip \labelsep {\bfseries #1:}]
	}
	{
	\end{trivlist}
}

%%%%%%%%%%%%%%%%%%%%%%%%%%%%%%%%%%%%%%%%%%%%%%%%%%
%%%%%%%%%%%%%%%%%%%%%%%%%%%%%%%%%%%%%%%%%%%%%%%%%%
%%%%%%%%%%%%%%%%%%%%%%%%%%%%%%%%%%%%%%%%%%%%%%%%%%
%
%
%    You need only modify code below this block.
%
%
%%%%%%%%%%%%%%%%%%%%%%%%%%%%%%%%%%%%%%%%%%%%%%%%%%
%%%%%%%%%%%%%%%%%%%%%%%%%%%%%%%%%%%%%%%%%%%%%%%%%%
%%%%%%%%%%%%%%%%%%%%%%%%%%%%%%%%%%%%%%%%%%%%%%%%%%
%
\title{Assignment: Problem Set X/Writing Assignment Y} %Change this to the assignment you are submitting.
\author{Name: First Last} %Change this to your name.
\date{Due Date: 00/00/2018 } %Change this to the due date for the assignment you are submitting.
\begin{document}
	\maketitle
	\thispagestyle{empty}
	
	\section*{List Your Collaborators:}%Enter your collaborators names below. Do not delete extra rows.
	
	\begin{itemize}
		\begin{framed}
			\item 
			Problem 1: Collaborator 1, Collaborator 2, etc.
			\\\\
		\end{framed}
		\begin{framed}
			\item 
			Problem 2: Collaborator 1, Collaborator 2, etc.
			\\\\
		\end{framed}
		\begin{framed}
			\item 
			Problem 3: Collaborator 1, Collaborator 2, etc. 
			\\\\
		\end{framed}
		\begin{framed}
			\item 
			Problem 4: Collaborator 1, Collaborator 2, etc.
			\\\\
		\end{framed}
		\begin{framed}
			\item 
			Problem 5: Collaborator 1, Collaborator 2, etc.
			\\\\
		\end{framed}
		\begin{framed}
			\item 
			Problem 6: Collaborator 1, Collaborator 2, etc.
			\\\\
		\end{framed}
	\end{itemize}
\newpage
%
%%%%%%%%%%%%%%%
%
% Your problem statements and solutions start here.
% Use the \newpage command between problems so that
% each of your problems begins on its own page.
%
%%%%%%%%%%%%%%%
%Provide the problem statement.
\begin{problem}{7}
	How do we use {\LaTeX} to typeset vectors, or systems of equations?
\end{problem}
%
\begin{solution}
	There a few ways to typeset math. Let's start with a simple vector. For example, we can define a vector in-line by: \verb|\(\vec{v}=\langle \frac{2}{3},-3,1\rangle\)| whose output looks like: \(\vec{v}=\langle \frac{2}{3},-3,1\rangle\), in the same line as the text.
	
	If, instead, we wanted this expression on its own line, we may use the code\\
	\verb|\[\vec{v}=\Big\langle \frac{2}{3},-3,1\Big\rangle\]|\\
	which results in
	\[
	\vec{v}=\Bigg\langle \frac{2}{3},-3,1\Bigg\rangle.
	\]
	
	Alternatively, if we want our equation to be numbered, we can use the \verb|align| environment:
	\begin{align}
		\vec{v}=\Bigg\langle \frac{2}{3},-3,1\Bigg\rangle
	\end{align}
	
	Google is your friend, there are a lot of resources available to help answer questions about \LaTeX.
	
	If we want out vectors to be aligned vertically, we might use the \verb|pmatrix| environment:
	\[
	\begin{pmatrix} 2 \\ -3 \\ 1 \end{pmatrix}
	\]
	
	The \verb|align| environment is useful when we have equations or computations that span several lines. For example:
	\begin{align}
	2x+3y-z &= 12\\
	x-y-12z &= 0\\
	12x+y+z &= 5
	\end{align}
	The \verb|&| symbol sets an alignment marker. You can mess around with these to arrange the parts of your align environment. You can use the \verb|align*| environment to omit equation numbers:
	\begin{align*}
	2x+3y-z &= 12 & \text{(the first equation)}\\
	x-y-12z &= 0\\
	12x+y+z &= 5
	\end{align*}
	Or, you can use the \verb|\nonumber| command on individual lines to suppress numbering.
	\begin{align}
	2x+3y-z &= 12\\
	x-y-12z &= 0\\
	12x+y+z &= 5 & \text{(the third equation has no number)} \nonumber
	\end{align}
\end{solution}

\newpage
\begin{problem}{13}
	What if we want to write a bunch of matrices?
\end{problem}
\begin{solution}
	There are many variations of the matrix environment, each of which must live inside a math environment such as \verb|\(\)| or \verb|\[\]| or \verb|align|.
	\[
	\begin{matrix}
		\alpha& \beta^{*}\\
		\gamma^{*}& \delta
	\end{matrix}
	\]

	A matrix, in line: \(
	\begin{bmatrix}
		\alpha&     \beta^{*}\\
		\gamma^{*}& \delta
	\end{bmatrix}
	\)

	\[	
	\begin{Bmatrix}
		\alpha&     \beta^{*}\\
		\gamma^{*}& \delta
	\end{Bmatrix}
	\]
	
	\[
	\begin{pmatrix}
		\alpha&     \beta^{*}\\
		\gamma^{*}& \delta
	\end{pmatrix}
	\]
	
	\[
	\begin{vmatrix}
		\alpha&     \beta^{*}\\
		\gamma^{*}& \delta
	\end{vmatrix}
	\]
	
	\[
	\begin{Vmatrix}
		\alpha&     \beta^{*}\\
		\gamma^{*}& \delta
	\end{Vmatrix}
	\]
	\[
	\begin{smallmatrix}
		\alpha&     \beta^{*}\\
		\gamma^{*}& \delta
	\end{smallmatrix}
	\]
	
	\begin{align}
		\begin{bmatrix}
			\alpha&     \beta^{*} & 7\\
			\gamma^{*}& \delta & -i
		\end{bmatrix}
		\times
		\begin{bmatrix}
		0&1\\
		-1& 1\\
		2 & 4
		\end{bmatrix}
	\end{align}
\end{solution}

\end{document}