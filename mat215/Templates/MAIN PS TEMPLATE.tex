\documentclass[12pt]{article}
\usepackage{latexsym, amssymb, amsmath, amsfonts, amscd, amsthm, xcolor, pgfplots}
\usepackage{framed}
\usepackage[margin=1in]{geometry}
\linespread{1} %Change the line spacing only if instructed to do so.

\newenvironment{problem}[2][Problem]
{
	\begin{trivlist} 
		\item[\hskip \labelsep {\bfseries #1 #2:}]
	}
{
	\end{trivlist}
	}

\newenvironment{solution}[1][Solution]
{
	\begin{trivlist} 
		\item[\hskip \labelsep {\itshape #1:}]
	}
	{
	\end{trivlist}
}

\newenvironment{collaborators}[1][Collaborator(s)]
{
	\begin{trivlist} 
		\item[\hskip \labelsep {\bfseries #1:}]
	}
	{
	\end{trivlist}
}

%%%%%%%%%%%%%%%%%%%%%%%%%%%%%%%%%%%%%%%%%%%%%%%%%%
%%%%%%%%%%%%%%%%%%%%%%%%%%%%%%%%%%%%%%%%%%%%%%%%%%
%%%%%%%%%%%%%%%%%%%%%%%%%%%%%%%%%%%%%%%%%%%%%%%%%%
%
%
%    You need only modify code below this block.
%
%
%%%%%%%%%%%%%%%%%%%%%%%%%%%%%%%%%%%%%%%%%%%%%%%%%%
%%%%%%%%%%%%%%%%%%%%%%%%%%%%%%%%%%%%%%%%%%%%%%%%%%
%%%%%%%%%%%%%%%%%%%%%%%%%%%%%%%%%%%%%%%%%%%%%%%%%%
%
\title{Assignment: Problem Set X} %Change this to the assignment you are submitting.
\author{Name: Oleksandr Yardas} %Change this to your name.
\date{Due Date: 00/00/2018 } %Change this to the due date for the assignment you are submitting.
\begin{document}
	\maketitle
	\thispagestyle{empty}
	
	\section*{List Your Collaborators:}%Enter your collaborators names below. Do not delete extra rows.
	
	\begin{itemize}
		\begin{framed}
			\item 
			Problem 1: None
			\\\\
		\end{framed}
		\begin{framed}
			\item 
			Problem 2: None
			\\\\
		\end{framed}
		\begin{framed}
			\item 
			Problem 3: None
			\\\\
		\end{framed}
		\begin{framed}
			\item 
			Problem 4: None
			\\\\
		\end{framed}
		\begin{framed}
			\item 
			Problem 5: None/Not Applicable
			\\\\
		\end{framed}
		\begin{framed}
			\item 
			Problem 6: None/Not Applicable
			\\\\
		\end{framed}
	\end{itemize}
\newpage
%
%%%%%%%%%%%%%%%
%
% Your problem statements and solutions start here.
% Use the \newpage command between problems so that
% each of your problems begins on its own page.
%
%%%%%%%%%%%%%%%

%FORMATTING OPTIONS
%FOR BLANK SPACES: \underline{\hspace{2cm}}
%FOR SPACES IN align OR SIMILAR ENVIRONMENTS:  \hphantom{1000}
%FOR MATRICES: \begin{matrix} \end{matrix}, can add p, b, B, v, V, small as suffix to "matrix"
%SETS: \mathbb{R}^, :\mathbb{R}^ \to \mathbb{R}^
%Vectors: \vec{},
%SUBSCRIPTS: _{}
%FRACTIONS: \frac{}{}
%FANCY LETTERS: \mathcal{}

%Provide the problem statement.
\begin{problem}{1}
->problem statement<-
\noindent
\newline
\newline
%a. [PART A STUFF]
\begin{solution}
-> solution <-
\end{solution}
%\vfill
%\centerline{PAGE 1 OF X FOR PROBLEM 1}\end{problem}
\end{problem}






\newpage
\begin{problem}{2}
->problem statement<-
\noindent
\newline
\newline
%a. [PART A STUFF]
\begin{solution}
-> solution <-
\end{solution}
%\vfill
%\centerline{PAGE 1 OF X FOR PROBLEM 2}
\end{problem}






\newpage
\begin{problem}{3}
->problem statement<-
\noindent
\newline
\newline
%a. [PART A STUFF]
\begin{solution}
-> solution <-
\end{solution}
%\vfill
%\centerline{PAGE 1 OF X FOR PROBLEM 3}
\end{problem}






\newpage
\begin{problem}{4}
->problem statement<-
\noindent
\newline
\newline
%a. [PART A STUFF]
\begin{solution}
-> solution <-
\end{solution}
%\vfill
%\centerline{PAGE 1 OF X FOR PROBLEM 4}
\end{problem}






\newpage
\begin{problem}{5}
->problem statement<-
\noindent
\newline
\newline
%a. [PART A STUFF]
\begin{solution}
-> solution <-
\end{solution}
%\vfill
%\centerline{PAGE 1 OF X FOR PROBLEM 5}
\end{problem}






\newpage
\begin{problem}{6}
->problem statement<-
\noindent
\newline
\newline
%a. [PART A STUFF]
\begin{solution}
-> solution <-
\end{solution}
%\vfill
%\centerline{PAGE 1 OF X FOR PROBLEM 6}
\end{problem}


\end{document}