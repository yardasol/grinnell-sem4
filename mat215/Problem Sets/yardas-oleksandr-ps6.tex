\documentclass[12pt]{article}
\usepackage{latexsym, amssymb, amsmath, amsfonts, amscd, amsthm, xcolor}
\usepackage{framed}
\usepackage[margin=1in]{geometry}
\linespread{1} %Change the line spacing only if instructed to do so.

\newenvironment{problem}[2][Problem]
{
	\begin{trivlist} 
		\item[\hskip \labelsep {\bfseries #1 #2:}]
	}
{
	\end{trivlist}
	}

\newenvironment{solution}[1][Solution]
{
	\begin{trivlist} 
		\item[\hskip \labelsep {\itshape #1:}]
	}
	{
	\end{trivlist}
}

\newenvironment{collaborators}[1][Collaborator(s)]
{
	\begin{trivlist} 
		\item[\hskip \labelsep {\bfseries #1:}]
	}
	{
	\end{trivlist}
}

%%%%%%%%%%%%%%%%%%%%%%%%%%%%%%%%%%%%%%%%%%%%%%%%%%
%%%%%%%%%%%%%%%%%%%%%%%%%%%%%%%%%%%%%%%%%%%%%%%%%%
%%%%%%%%%%%%%%%%%%%%%%%%%%%%%%%%%%%%%%%%%%%%%%%%%%
%
%
%    You need only modify code below this block.
%
%
%%%%%%%%%%%%%%%%%%%%%%%%%%%%%%%%%%%%%%%%%%%%%%%%%%
%%%%%%%%%%%%%%%%%%%%%%%%%%%%%%%%%%%%%%%%%%%%%%%%%%
%%%%%%%%%%%%%%%%%%%%%%%%%%%%%%%%%%%%%%%%%%%%%%%%%%
%
\title{Assignment: Problem Set 6} %Change this to the assignment you are submitting.
\author{Name: Oleksandr Yardas} %Change this to your name.
\date{Due Date: 02/19/2018 } %Change this to the due date for the assignment you are submitting.
\begin{document}
	\maketitle
	\thispagestyle{empty}
	
	\section*{List Your Collaborators:}%Enter your collaborators names below. Do not delete extra rows.
\[
{\bf Note:}
\]
\[
\text{In Problems 1 and 4, I have written in a different
color (blue) %\textcolor{red}{c}\textcolor{cyan}{o}\textcolor{brown}{l}\textcolor{magenta}{o}\textcolor{blue}{r}
the parts that go into the blanks.}
\]
	\begin{itemize}
		\begin{framed}
			\item 
			Problem 1: None
			\\\\
		\end{framed}
		\begin{framed}
			\item 
			Problem 2: None
			\\\\
		\end{framed}
		\begin{framed}
			\item 
			Problem 3: None
			\\\\
		\end{framed}
		\begin{framed}
			\item 
			Problem 4: None
			\\\\
		\end{framed}
		\begin{framed}
			\item 
			Problem 5: Not Applicable
			\\\\
		\end{framed}
		\begin{framed}
			\item 
			Problem 6: Not Applicable
			\\\\
		\end{framed}
	\end{itemize}
\newpage
%
%%%%%%%%%%%%%%%
%
% Your problem statements and solutions start here.
% Use the \newpage command between problems so that
% each of your problems begins on its own page.
%
%%%%%%%%%%%%%%%
%Provide the problem statement.
\begin{problem}{1}
Fill in the blanks below with appropriate phrases so that the result is a correct proof of the following statement: If $\vec{u}, \vec{w} \in \mathbb{R}$ and $ \vec{w} \in$ Span$(\vec{u})$, then Span$(\vec{w}) \in$ Span$(\vec{u})$.
%\newline
%\newline
%\noindent
%Let $\vec{v} \in$ Span$(\vec{w})$ be arbitrary. Since $\vec{w} \in$ Span$(\vec{u})$, we can \underline{\hspace{2cm}}. Since $\vec{v} \in$ Span$(\vec{w})$, we can \underline{\hspace{2cm}}. Now notice that $\vec{v}=$\underline{\hspace{2cm}}. Since \underline{\hspace{2cm}}$\in \mathbb{R}$, we conclude that $\vec{v} \in$ Span$(\vec{u})$. Since $\vec{v} \in$ Span$(\vec{u})$ was arbitrary, the result follows.
\[
{\bf Note:}
\]
\[
\text{I have written in a different
color (blue) %\textcolor{red}{c}\textcolor{cyan}{o}\textcolor{brown}{l}\textcolor{magenta}{o}\textcolor{blue}{r}
the parts that go into the blanks.}
\]

\begin{solution}
Let $\vec{v} \in$ Span$(\vec{w})$ be arbitrary. Since $\vec{w} \in$ Span$(\vec{u})$, we can \textcolor{blue}{fix a $c\in\mathbb{R}$ with $\vec{w}=c \cdot \vec{u}$ (by definition of Span$(\vec{u})$)}. Since $\vec{v} \in$ Span$(\vec{w})$, we can \textcolor{blue}{fix a $d\in\mathbb{R}$ with $\vec{v}=d \cdot \vec{w}$ (by definition of Span$(\vec{w})$)}. Now notice that $\vec{v}=$\textcolor{blue}{$d \cdot \vec{w} = d \cdot (c \cdot \vec{u}) = (cd) \cdot \vec{u}$ (by Proposition 2.2.1.9)}. Since \textcolor{blue}{$cd$} $\in \mathbb{R}$, we conclude that $\vec{v} \in$ Span$(\vec{u})$. Since $\vec{v} \in$ Span$(\vec{u})$ was arbitrary, the result follows.
\end{solution}
\end{problem}






\newpage
\begin{problem}{2}
Given $\vec{u} \in \mathbb{R}^2$, is the set Span$(\vec{u})$ always closed under componentwise multiplication? In other words, if
\[
\begin{pmatrix} a_{1}\\b_{1} \end{pmatrix} \in \text{ Span}(\vec{u}) \hphantom{1000} \text{ and } \hphantom{1000} \begin{pmatrix} a_{2}\\b_{2} \end{pmatrix} \in \text{ Span}(\vec{u})\text{,}
\]
must it be the case that
\[
\begin{pmatrix} a_{1} a_{2}\\b_{1} b_{2} \end{pmatrix} \in \text{ Span}(\vec{u}) \text{?}
\]
Either argue that this is always true, or provide a specific counterexample (with justification).
\begin{solution}
Let  $\begin{pmatrix} a_{1}\\b_{1} \end{pmatrix}=\vec{v_{1}},\begin{pmatrix} a_{2}\\b_{2} \end{pmatrix}=\vec{v_{2}}$. Consider the case in which $\vec{u}=\begin{pmatrix} 1\\2 \end{pmatrix}$, $\vec{v_{1}}=\begin{pmatrix} 3\\6 \end{pmatrix}$, $\vec{v_{2}}=\begin{pmatrix} 4\\8 \end{pmatrix}$. Note that
\[
3 \cdot \vec{u} = 3 \cdot \begin{pmatrix} 1\\2 \end{pmatrix} = \begin{pmatrix} 3 \cdot 1\\ 3 \cdot 2 \end{pmatrix} = \begin{pmatrix} 3\\6 \end{pmatrix} = \vec{v_{1}}
\]
and
\[
4 \cdot \vec{u} = 4 \cdot \begin{pmatrix} 1\\2 \end{pmatrix} = \begin{pmatrix} 4 \cdot 1\\ 4 \cdot 2 \end{pmatrix} = \begin{pmatrix} 4\\8 \end{pmatrix} = \vec{v_{2}}
\]
So $\vec{v_{1}}, \vec{v_{2}} \in$ Span$(\vec{u})$ by definition of Span$(\vec{u})$. If we take the componentwise product of $\vec{v_{1}}, \vec{v_{2}}$, we get:
\begin{align*}
\begin{pmatrix} 3\cdot4\\6\cdot8 \end{pmatrix} =& \begin{pmatrix} 12\\48 \end{pmatrix}\\
=& \begin{pmatrix} 12 \cdot (1)\\24 \cdot(2) \end{pmatrix}
\end{align*}
This vector cannot be represented by the product of a scalar and $\vec{u}$. We have found a $\vec{v_{1}},\vec{v_{2}} \in$ Span$(\vec{u})$ such that their componentwise product is not an element of Span$(\vec{u})$, therefore it is not the case that the set Span$(\vec{u})$ always closed under componentwise multiplication.
\end{solution}
\end{problem}






\newpage
\begin{problem}{3}
Let $\vec{u_{1}} = \begin{pmatrix}-1 \\ 2 \end{pmatrix}$, let $\vec{u_{2}} = \begin{pmatrix}5 \\ 1 \end{pmatrix}$, and let $\alpha = (\vec{u_{1}}, \vec{u_{2}})$. In each part, briefly explain how you carried out your computation. 
\noindent
\newline
\newline
a. Show that Span$(\vec{u_{1}}, \vec{u_{2}})= \mathbb{R}^2$, so $\alpha=(\vec{u_{1}}, \vec{u_{2}})$ is a basis for $\mathbb{R}^2$.
\begin{solution}
Let $\vec{v} \in \mathbb{R}^2$ be arbitrary. %We want to show that Span$(\vec{u_{1}}, \vec{u_{2}})= \mathbb{R}^2$, so we do a double containment proof, that is, we show that Span$(\vec{u_{1}}, \vec{u_{2}}) \subseteq \mathbb{R}^2$ and $\mathbb{R}^2 \subseteq$ Span$(\vec{u_{1}}, \vec{u_{2}})$. We know that Span$(\vec{u_{1}}, \vec{u_{2}}) \subseteq \mathbb{R}^2$ by definition. So we just need to show that  $\mathbb{R}^2 \subseteq$ Span$(\vec{u_{1}}, \vec{u_{2}})$. 
We define Span$(\vec{u_{1}}, \vec{u_{2}})$ as
\[
\text{Span}(\vec{u_{1}}, \vec{u_{2}}) = \{c_{1}\vec{u_{1}} + c_{2}\vec{u_{2}} : c_{1},c_{2} \in \mathbb{R}\}\text{,}
\]
which we can rewrite as
\[
\text{Span}(\vec{u_{1}}, \vec{u_{2}}) = \{\vec{v} \in \mathbb{R}^2: \text{There exist } c_{1},c_{2} \in \mathbb{R} \text{ with } \vec{v}=c_{1}\vec{u_{1}} + c_{2}\vec{u_{2}}\}\text{.} 
\]
Now, notice that
\[
((-1) \cdot 1) - (5 \cdot 2) = -1 - 10 = -11 \neq 0
\]
By Theorem 2.3.10, it follows that for every $\vec{v} \in \mathbb{R}^2$, there does indeed exist $c_{1},c_{2} \in \mathbb{R}$ with $\vec{v}=c_{1}\vec{u_{1}} + c_{2}\vec{u_{2}}$ for our particular values of $\vec{u_{1}},\vec{u_{2}}$. This means that the rule by which $\vec{v}$ is carved out of $\mathbb{R}^2$ to construct  Span$(\vec{u_{1}}, \vec{u_{2}})$ is true for all $\vec{v} \in \mathbb{R}^2$, and it follows that
\[
\text{Span}(\vec{u_{1}}, \vec{u_{2}})=\{\vec{v} : \vec{v} \in \mathbb{R}^2\}\text{,}
\]
which is just the set of all vectors $\vec{v}$ for all $\vec{v} \in \mathbb{R}^2$. But the set of all vectors $\vec{v} \in \mathbb{R}^2$ is just the set $\mathbb{R}^2$, therefore, Span$(\vec{u_{1}}, \vec{u_{2}}) = \mathbb{R}^2$. By definition 2.3.11 that $\alpha=(\vec{u_{1}}, \vec{u_{2}})$ is a basis for $\mathbb{R}^2$.

\end{solution}
\noindent
\newline
\newline
b. Find the coordinates of $\begin{pmatrix}5 \\ 1 \end{pmatrix}$ relative to $\alpha$. In other words, calculate $Coord_{\alpha}\left(  \begin{pmatrix}5 \\ 1 \end{pmatrix} \right)$.
\begin{solution}
By the Proposition 2.3.13, we have 
\[
Coord_{\alpha}\left(\begin{pmatrix}j\\k\end{pmatrix}\right) = \frac{1}{(-11)} \cdot \begin{pmatrix}(1)\cdot j-(5)\cdot k\\(-1) \cdot k-(2) \cdot j\end{pmatrix}
\]
So we have
\begin{align*}
Coord_{\alpha}\left(  \begin{pmatrix}5 \\ 1 \end{pmatrix} \right) =& \frac{1}{(-11)} \cdot \begin{pmatrix}(1)\cdot (5)-(5)\cdot (1)\\(-1) \cdot (1)-(2) \cdot (5)\end{pmatrix}\\
=& \frac{1}{(-11)} \cdot \begin{pmatrix}5-5\\-1-10\end{pmatrix}\\
=& \frac{1}{(-11)} \cdot \begin{pmatrix}0\\-11\end{pmatrix}\\
=& \begin{pmatrix} \frac{1}{(-11)} \cdot 0\\ \frac{1}{(-11)} \cdot (-11)\end{pmatrix}\\
=& \begin{pmatrix}0\\1\end{pmatrix}
\end{align*}
So the coordinates of $\begin{pmatrix}5 \\ 1 \end{pmatrix}$ relative to $\alpha$ are $\begin{pmatrix}0\\1\end{pmatrix}$.
\[
\text{PAGE 1 OF 2 FOR PROBLEM 3}
\]
\end{solution}
\end{problem}






\newpage
\begin{problem}{4}
In this problem we work through the proof of Proposition 2.3.8 in the notes, which says the following: Let $\vec{u_{1}}, \vec{u_{2}} \in \mathbb{R}^2$. The following are equivalent.
\begin{enumerate}
\item Span$(\vec{u_{1}}, \vec{u_{2}}) =$ Span$(\vec{u_{1}})$.
\item $\vec{u_{2}} \in$ Span$(\vec{u_{1}})$.
\end{enumerate}
Fill in the blanks below with appropriate phrases so that the result is a correct proof:
%\noindent
%\newline
%\newline
%We first show that 1 implies 2. Assume that 1 is true, so assume that Span$(\vec{u_{1}}, \vec{u_{2}}) =$ Span$(\vec{u_{1}})$. Notice that $\vec{u_{2}}=$\underline{\hspace{2cm}}. Since \underline{\hspace{2cm}}$\in \mathbb{R}$, it follows that $\vec{u_{2}} \in$ Span$(\vec{u_{1}}, \vec{u_{2}})$. Since Span$(\vec{u_{1}}, \vec{u_{2}})=$ Span$(\vec{u_{1}})$, we conclude that \underline{\hspace{2cm}}.
%\noindent
%\newline
%\newline
%We now show that 2 implies 1. Assume then that 2 is true, so assume that $\vec{u_{2}} \in$ Span$(\vec{u_{1}})$. By definition, we can \underline{\hspace{2cm}}. To show that Span$(\vec{u_{1}}, \vec{u_{2}})=$Span$(\vec{u_{1}})$, we give a double containment proof.
%\begin{itemize}
%\item Using Proposition \underline{\hspace{2cm}}, we know immediately that Span$(\vec{u_{1}}) \subseteq$ Span$(\vec{u_{1}}, \vec{u_{2}})$.
%\item We now show that Span$(\vec{u_{1}}, \vec{u_{2}}) \subseteq$ Span$(\vec{u_{1}})$. Let $\vec{v} \in$ Span$(\vec{u_{1}},\vec{u_{2}}))$ be arbitrary. By definition we can \underline{\hspace{2cm}}. Notice that $\vec{v} =$\underline{\hspace{2cm}}. Since  \underline{\hspace{2cm}}$\in \mathbb{R}$, it follows that $\vec{v} \in$ Span$(\vec{u_{1}})$. Since $\vec{v} \in$ Span$(\vec{u_{1}}, \vec{u_{2}})$ was arbitrary, we conclude that Span$(\vec{u_{1}}, \vec{u_{2}}) \subseteq$ Span$(\vec{u_{1}})$.
%\end{itemize}
%\noindent
%Since we have shown both Span$(\vec{u_{1}}) \subseteq$ Span$(\vec{u_{1}}, \vec{u_{2}})$ and Span$(\vec{u_{1}}, \vec{u_{2}}) \subseteq$ Span$(\vec{u_{1}})$, we conclude that Span$(\vec{u_{1}}, \vec{u_{2}}) =$ Span$(\vec{u_{1}})$.
\[
{\bf Note:}
\]
\[
\text{I have written in a different
color (blue) %\textcolor{red}{c}\textcolor{cyan}{o}\textcolor{brown}{l}\textcolor{magenta}{o}\textcolor{blue}{r}
the parts that go into the blanks.}
\]

\begin{solution}
We first show that 1 implies 2. Assume that 1 is true, so assume that Span$(\vec{u_{1}}, \vec{u_{2}}) =$ Span$(\vec{u_{1}})$. Notice that $\vec{u_{2}}= \textcolor{blue}{0\cdot \vec{u_{1}}+c_{2} \vec{u_{2}} = \vec{0} + c_{2} \vec{u_{2}} = c_{2} \vec{u_{2}}}$. Since $\textcolor{blue}{0,c_{2}} \in \mathbb{R}$, it follows that $\vec{u_{2}} \in$ Span$(\vec{u_{1}}, \vec{u_{2}})$. Since Span$(\vec{u_{1}}, \vec{u_{2}})=$ Span$(\vec{u_{1}})$, we conclude that \textcolor{blue}{$\vec{u_{2}} \in$ Span$(\vec{u_{1}})$}.
\noindent
\newline
\newline
We now show that 2 implies 1. Assume then that 2 is true, so assume that $\vec{u_{2}} \in$ Span$(\vec{u_{1}})$. By definition, we can \textcolor{blue}{fix a $c \in \mathbb{R}$ with $\vec{u_{2}}= c \cdot \vec{u_{1}}$}. To show that Span$(\vec{u_{1}}, \vec{u_{2}})=$ Span$(\vec{u_{1}})$, we give a double containment proof.
\begin{itemize}
\item Using Proposition \textcolor{blue}{2.3.7}, we know immediately that Span$(\vec{u_{1}}) \subseteq$ Span$(\vec{u_{1}}, \vec{u_{2}})$.
\item We now show that Span$(\vec{u_{1}}, \vec{u_{2}}) \subseteq$ Span$(\vec{u_{1}})$. Let $\vec{v} \in$ Span$(\vec{u_{1}},\vec{u_{2}})$ be arbitrary. By definition we can \textcolor{blue}{fix a $c_{1},c_{2} \in \mathbb{R}$ with $\vec{v}=c_{1}\vec{u_{1}} +c_{2}\vec{u_{2}}$}. Notice that $\vec{v} = $\textcolor{blue}{$c_{1}\vec{u_{1}} +c_{2}(c \cdot \vec{u_{1}}) = c_{1}\vec{u_{1}} +(c_{2}c) \cdot \vec{u_{1}} = (c_{1}c_{2}c) \cdot \vec{u_{1}}$}. Since  \textcolor{blue}{$c_{1}c_{2}c$} $\in \mathbb{R}$, it follows that $\vec{v} \in$ Span$(\vec{u_{1}})$. Since $\vec{v} \in$ Span$(\vec{u_{1}}, \vec{u_{2}})$ was arbitrary, we conclude that Span$(\vec{u_{1}}, \vec{u_{2}}) \subseteq$ Span$(\vec{u_{1}})$.
\end{itemize}
\noindent
Since we have shown both Span$(\vec{u_{1}}) \subseteq$ Span$(\vec{u_{1}}, \vec{u_{2}})$ and Span$(\vec{u_{1}}, \vec{u_{2}}) \subseteq$ Span$(\vec{u_{1}})$, we conclude that Span$(\vec{u_{1}}, \vec{u_{2}}) =$ Span$(\vec{u_{1}})$.


\end{solution}
\newpage
c. Find the coordinates of $\begin{pmatrix}8 \\ 17 \end{pmatrix}$ relative to $\alpha$. In other words, calculate $Coord_{\alpha}\left(  \begin{pmatrix}8 \\ 17 \end{pmatrix} \right)$.
\begin{solution}
By the Proposition 2.3.13, we have 
\[
Coord_{\alpha}\left(\begin{pmatrix}j\\k\end{pmatrix}\right) = \frac{1}{(-11)} \cdot \begin{pmatrix}(1)\cdot j-(5)\cdot k\\(-1) \cdot k-(2) \cdot j\end{pmatrix}
\]
So we have
\begin{align*}
Coord_{\alpha}\left(  \begin{pmatrix}5 \\ 1 \end{pmatrix} \right) =& \frac{1}{(-11)} \cdot \begin{pmatrix}(1)\cdot (8)-(5)\cdot (17)\\(-1) \cdot (17)-(2) \cdot (8)\end{pmatrix}\\
=& \frac{1}{(-11)} \cdot \begin{pmatrix}8-85\\-17-16\end{pmatrix}\\
=& \frac{1}{(-11)} \cdot \begin{pmatrix}-93\\-33\end{pmatrix}\\
=& \begin{pmatrix} \frac{1}{(-11)} \cdot (-93)\\ \frac{1}{(-11)} \cdot (-33)\end{pmatrix}\\
=& \begin{pmatrix}\frac{93}{11}\\3\end{pmatrix}
\end{align*}
So the coordinates of $\begin{pmatrix}8 \\ 17 \end{pmatrix}$ relative to $\alpha$ are $\begin{pmatrix}\frac{93}{11}\\3\end{pmatrix}$.
\end{solution}
\noindent
\newline
\newline
\newline
\newline
\newline
\newline
\newline
\newline
\newline
\newline
\newline
\newline
\newline
\newline
\newline
\newline
\newline
\newline
\newline
\newline
\newline
\newline
\[
\text{PAGE 2 OF 2 FOR PROBLEM 3}
\]

\end{problem}


\end{document}