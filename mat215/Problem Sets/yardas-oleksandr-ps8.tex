\documentclass[12pt]{article}
\usepackage{latexsym, amssymb, amsmath, amsfonts, amscd, amsthm}
\usepackage{framed}
\usepackage[margin=1in]{geometry}
\linespread{1} %Change the line spacing only if instructed to do so.

\newenvironment{problem}[2][Problem]
{
	\begin{trivlist} 
		\item[\hskip \labelsep {\bfseries #1 #2:}]
	}
{
	\end{trivlist}
	}

\newenvironment{solution}[1][Solution]
{
	\begin{trivlist} 
		\item[\hskip \labelsep {\itshape #1:}]
	}
	{
	\end{trivlist}
}

\newenvironment{collaborators}[1][Collaborator(s)]
{
	\begin{trivlist} 
		\item[\hskip \labelsep {\bfseries #1:}]
	}
	{
	\end{trivlist}
}

%%%%%%%%%%%%%%%%%%%%%%%%%%%%%%%%%%%%%%%%%%%%%%%%%%
%%%%%%%%%%%%%%%%%%%%%%%%%%%%%%%%%%%%%%%%%%%%%%%%%%
%%%%%%%%%%%%%%%%%%%%%%%%%%%%%%%%%%%%%%%%%%%%%%%%%%
%
%
%    You need only modify code below this block.
%
%
%%%%%%%%%%%%%%%%%%%%%%%%%%%%%%%%%%%%%%%%%%%%%%%%%%
%%%%%%%%%%%%%%%%%%%%%%%%%%%%%%%%%%%%%%%%%%%%%%%%%%
%%%%%%%%%%%%%%%%%%%%%%%%%%%%%%%%%%%%%%%%%%%%%%%%%%
%
\title{Assignment: Problem Set 8} %Change this to the assignment you are submitting.
\author{Name: Oleksandr Yardas} %Change this to your name.
\date{Due Date: 02/26/2018 } %Change this to the due date for the assignment you are submitting.
\begin{document}
	\maketitle
	\thispagestyle{empty}
	
	\section*{List Your Collaborators:}%Enter your collaborators names below. Do not delete extra rows.
	
	\begin{itemize}
		\begin{framed}
			\item 
			Problem 1: None
			\\\\
		\end{framed}
		\begin{framed}
			\item 
			Problem 2: None
			\\\\
		\end{framed}
		\begin{framed}
			\item 
			Problem 3: None
			\\\\
		\end{framed}
		\begin{framed}
			\item 
			Problem 4: None
			\\\\
		\end{framed}
		\begin{framed}
			\item 
			Problem 5: None
			\\\\
		\end{framed}
		\begin{framed}
			\item 
			Problem 6: None
			\\\\
		\end{framed}
	\end{itemize}
\newpage
%
%%%%%%%%%%%%%%%
%
% Your problem statements and solutions start here.
% Use the \newpage command between problems so that
% each of your problems begins on its own page.
%
%%%%%%%%%%%%%%%
%Provide the problem statement.
\begin{problem}{1}
Consider the unique linear transformation $T:\mathbb{R}^2 \to \mathbb{R}^2$ such that
\begin{align*}
T\left(\begin{pmatrix}9\\4\end{pmatrix}\right)=\begin{pmatrix}1\\-5\end{pmatrix} && \text{ and } && T\left(\begin{pmatrix}2\\1\end{pmatrix}\right)=\begin{pmatrix}-2\\3\end{pmatrix}\text{.}
\end{align*}
Determine, with explanation, the value of
\[
T\left(\begin{pmatrix}6\\2\end{pmatrix}\right)\text{.}
\]
\noindent
\newline
\newline

\begin{solution}
Notice that:
\begin{align*}
\begin{pmatrix}6\\2\end{pmatrix}=&\begin{pmatrix}(18-12)\\(8-6)\end{pmatrix}\\
=& \begin{pmatrix}18\\8\end{pmatrix} - \begin{pmatrix}12\\6\end{pmatrix}\\
=& \begin{pmatrix}2\cdot 9\\2\cdot 4\end{pmatrix} - \begin{pmatrix}6 \cdot 2\\6 \cdot 1\end{pmatrix}\\
=& 2\cdot \begin{pmatrix}9\\4\end{pmatrix} - 6 \cdot \begin{pmatrix}2\\1\end{pmatrix}
\end{align*}
So $\begin{pmatrix}6\\2\end{pmatrix}=2\cdot \begin{pmatrix}9\\4\end{pmatrix} - 6 \cdot \begin{pmatrix}2\\1\end{pmatrix}$ and we have
\begin{align*}
T\left(\begin{pmatrix}6\\2\end{pmatrix}\right) =&T\left(2\cdot \begin{pmatrix}9\\4\end{pmatrix} - 6 \cdot \begin{pmatrix}2\\1\end{pmatrix}\right) &\\
=& T\left(2\cdot \begin{pmatrix}9\\4\end{pmatrix}\right) + T\left(-6 \cdot \begin{pmatrix}2\\1\end{pmatrix}\right) & \text{(By the definition of linear transformation)}\\
=& 2\cdot T\left(\begin{pmatrix}9\\4\end{pmatrix}\right) + (-6) \cdot T\left(\begin{pmatrix}2\\1\end{pmatrix}\right) & \text{(By the definition of linear transformation)}\\
=& 2\cdot \begin{pmatrix}1\\-5\end{pmatrix} + (-6) \cdot \begin{pmatrix}-2\\3\end{pmatrix} & \text{(By definition of $T$)}\\
=& \begin{pmatrix}2\cdot 1\\ 2\cdot (-5)\end{pmatrix} + \begin{pmatrix}(-6) \cdot (-2)\\(-6) \cdot 3\end{pmatrix} &\\
=& \begin{pmatrix}2\\ -10\end{pmatrix} + \begin{pmatrix}12\\-18\end{pmatrix} &\\
=& \begin{pmatrix}2 +12 \\ (-10) + (-18) \end{pmatrix} & \\
=& \begin{pmatrix} 14 \\ -28 \end{pmatrix} &
\end{align*}
Therefore, $T\left(\begin{pmatrix}6\\2\end{pmatrix}\right) =\begin{pmatrix} 14 \\ -28 \end{pmatrix}$.

\end{solution}
\end{problem}






\newpage
\begin{problem}{2}
Compute
\[
\begin{pmatrix}4 & 3 \\ -7 & 1\end{pmatrix} \begin{pmatrix}1\\5\end{pmatrix} \text{.}
\]
What does your answer mean in terms of linear transformations? Explain.
\noindent
\newline
\newline

\begin{solution}
By Definition 3.1.3, we have
\begin{align*}
\begin{pmatrix}4 & 3 \\ -7 & 1\end{pmatrix} \begin{pmatrix}1\\5\end{pmatrix} =& \begin{pmatrix}4 \cdot 1 + 3\cdot 5 \\ (-7) \cdot 1 +1 \cdot 5 \end{pmatrix}\\
=& \begin{pmatrix}4  + 15 \\ -7 + 5  \end{pmatrix}\\
=& \begin{pmatrix}19\\ -2\end{pmatrix}
\end{align*}
In terms of linear transformations, we can say that, for the unique linear transformation $T:\mathbb{R}^2 \to \mathbb{R}^2$ such that $T(\vec{e_{1}})=\begin{pmatrix} 4\\-7\end{pmatrix}$ and $T(\vec{e_{2}})=\begin{pmatrix} 3\\1\end{pmatrix}$ (where $\vec{e_{1}}=\begin{pmatrix} 1\\0\end{pmatrix},\vec{e_{2}}=\begin{pmatrix} 0\\1\end{pmatrix}$), we define $[T]=\begin{pmatrix}4 & 3 \\ -7 & 1\end{pmatrix}$ (by Definition 3.1.1) and so by Proposition 3.1.4, $T\left(\begin{pmatrix}1\\5\end{pmatrix}\right)=[T]\begin{pmatrix}1\\5\end{pmatrix}=\begin{pmatrix}4 & 3 \\ -7 & 1\end{pmatrix} \begin{pmatrix}1\\5\end{pmatrix}=\begin{pmatrix}19\\ -2\end{pmatrix}$.
\end{solution}
\end{problem}






\newpage
\begin{problem}{3}
Let $\theta \in \mathbb{R}$. Define $C_{\theta} : \mathbb{R}^2 \to \mathbb{R}^2$ by letting $C_{\theta}(\vec{v})$ be the result of rotating $\vec{v}$ {\it clockwise} around the origin by and angle of $\theta$. It can be shown geometrically that $C_[\theta]$ is a linear transformation (no need to do this). What is $[C_{\theta}]$? Explain your reasoning, and simplify your answer as much as possible.
\noindent
\newline
\newline
\begin{solution}
Let $\vec{e_{1}}=\begin{pmatrix} 1\\0\end{pmatrix},\vec{e_{2}}=\begin{pmatrix} 0\\1\end{pmatrix}$. We assume that $C_{\theta} : \mathbb{R}^2 \to \mathbb{R}^2$ is a linear transformation. Fix $a,b,c,d \in \mathbb{R}$ such that $C_{\theta}(\vec{e_{1}})=\begin{pmatrix}a\\c\end{pmatrix}$ and $C_{\theta}(\vec{e_{2}})=\begin{pmatrix}b\\d\end{pmatrix}$. By Definition 3.1.1, the standard matrix of $C_{\theta}$, denoted by $[C_{\theta}]$, is the matrix $\begin{pmatrix} a &b \\c&d\end{pmatrix}$. We defined $C_{\theta} : \mathbb{R}^2 \to \mathbb{R}^2$ by letting $C_{\theta}(\vec{v})$ be the result of rotating $\vec{v}$ {\it clockwise} around the origin by and angle of $\theta$. By convention, clockwise rotations have negative signs. From Calculus II, we know that
\begin{align*}
\begin{pmatrix}a\\c\end{pmatrix} =&\begin{pmatrix}\text{comp}_{\vec{e_{1}}}(C_{\theta}(\vec{e_{1}}))\\ \text{comp}_{\vec{e_{2}}}(C_{\theta}(\vec{e_{1}})) \end{pmatrix} &\\
=&\begin{pmatrix} \frac{\vec{e_{1}} \bullet C_{\theta}(\vec{e_{1}})}{\| \vec{e_{1}} \|} \\ \frac{\vec{e_{2}} \bullet C_{\theta}(\vec{e_{1}})}{\| \vec{e_{2}} \|} \end{pmatrix} &\text{(by definition of comp)} \\
=& \begin{pmatrix} \frac{\| \vec{e_{1}} \| \| C_{\theta}(\vec{e_{1}}) \| \cos (-\theta)}{\| \vec{e_{1}} \|} \\ \frac{\| \vec{e_{2}} \| \| C_{\theta}(\vec{e_{1}}) \| \cos(-\theta - \frac{\pi}{2})}{\| \vec{e_{2}} \|} \end{pmatrix} & \text{(by definition of dot product)}\\
=& \begin{pmatrix} \frac{1 \cdot 1 \cdot \cos (-\theta)}{1} \\ \frac{1 \cdot 1 \cdot \cos (-(\theta +\frac{\pi}{2}))}{1} \end{pmatrix} &\\
=& \begin{pmatrix} \cos (-\theta)\\ \cos (-(\theta +\frac{\pi}{2})) \end{pmatrix} & \\
=&  \begin{pmatrix} \cos (\theta)\\ \cos (\theta +\frac{\pi}{2}) \end{pmatrix} & \text{(by definition of even functions)}\\
=& \begin{pmatrix} \cos (\theta)\\ -\sin (\theta) \end{pmatrix} & \text{(by properties of trig functions)}
\end{align*}
\newline
\newline
\newline
\newline
\newline
\newline
\newline
\newline
\newline
\newline
\newline
\[
\text{PAGE 1 OF 2 FOR PROBLEM 3}
\]
\end{solution}
\end{problem}






\newpage
\begin{problem}{4}
For each of the following, consider the linear transformation $T: \mathbb{R}^2 \to \mathbb{R}^2$ that has the given matrix as its standard matrix. Describe the action of $T$ geometrically. It may help to plug in a few points and/or make some case distinctions.
\noindent
\newline
\newline
a. $\begin{pmatrix}1&0\\0&0\end{pmatrix}$
\begin{solution}
Given a vector $\begin{pmatrix}a\\b\end{pmatrix}$ with arbitrary $a,b \in \mathbb{R}$, $T$ takes $\begin{pmatrix}a\\b\end{pmatrix}$ as its input and gives $\begin{pmatrix}1&0\\0 & 0\end{pmatrix} \begin{pmatrix}a\\b\end{pmatrix}=\begin{pmatrix}1\cdot a+ 0\cdot b\\0\cdot a+0\cdot b\end{pmatrix} = \begin{pmatrix}a\\0\end{pmatrix}$ as its output. In other words, $T$ acts on an arbitrary vector $\vec{v} \in \mathbb{R}^2$ by taking the vector projection of $\vec{v}$ onto the $x$-axis. Geometrically this has the effect of ``flattening'' $\vec{v}$ onto the $x$-axis while constrained bewteen the $y$-axis and the line that intersects the terminal point of $\vec{v}$ that is parallel to the $y$-axis.
\end{solution}

\noindent
\newline
\newline
b. $\begin{pmatrix}k&0\\0&k\end{pmatrix}$ for a fixed $k \in \mathbb{R}$ with $k>0$.
\begin{solution}
Given a vector $\begin{pmatrix}a\\b\end{pmatrix}$ with arbitrary $a,b \in \mathbb{R}$, $T$ takes $\begin{pmatrix}a\\b\end{pmatrix}$ as its input and gives $\begin{pmatrix}k&0\\0 & k\end{pmatrix} \begin{pmatrix}a\\b\end{pmatrix}=\begin{pmatrix}k\cdot a+ 0\cdot b\\0\cdot a+k\cdot b\end{pmatrix} = \begin{pmatrix}ka\\kb\end{pmatrix} = k\cdot \begin{pmatrix}a\\b\end{pmatrix}$ as its output (for a fixed $k \in \mathbb{R}$ with $k>0$). In other words, $T$ acts on an arbitrary vector $\vec{v} \in \mathbb{R}^2$ by scaling $\vec{v}$ by a positive real number $k$. In the case where $k>1$, this has the effect of ``stretching'' $\vec{v}$ out away from the origin along its original direction. In the case where $k=1$, this has the effect of doing nothing ($\vec{v}$ does not change). In the case where $1>k>0$, this has the effect of ``squishing'' $\vec{v}$ in towards the origin along its original direction.
\end{solution}

\noindent
\newline
\newline
c. $\begin{pmatrix}k&0\\0&1\end{pmatrix}$ for a fixed $k \in \mathbb{R}$ with $k>0$.
\begin{solution}
Given a vector $\begin{pmatrix}a\\b\end{pmatrix}$ with arbitrary $a,b \in \mathbb{R}$, $T$ takes $\begin{pmatrix}a\\b\end{pmatrix}$ as its input and gives $\begin{pmatrix}k&0\\0 & 1\end{pmatrix} \begin{pmatrix}a\\b\end{pmatrix}=\begin{pmatrix}k\cdot a+ 0\cdot b\\0\cdot a+1\cdot b\end{pmatrix} = \begin{pmatrix}ka\\b\end{pmatrix}$ as its output (for a fixed $k \in \mathbb{R}$ with $k>0$). In other words, $T$ acts on an arbitrary vector $\vec{v} \in \mathbb{R}^2$ by taking the linear combination of the projection of $\vec{v}$ onto the $x$-axis scaled by some positive real number $k$ and the projection of $\vec{v}$ onto the $y$-axis. In the case where $k>1$,  this has the effect of ``stretching'' $\vec{v}$ out along the $x$-axis away from the $y$-axis while the distance of the terminal point of $\vec{v}$ away from the $x$-axis stays constant. In the case where $k=1$, this has the effect of doing nothing ($\vec{v}$ does not change). In the case where $1>k>0$, this has the effect of ``squishing'' $\vec{v}$ inwards along the $x$-axis towards the $y$-axis while the distance of the terminal point of $\vec{v}$ away from the $x$-axis stays constant.
\end{solution}

%\newline
%\newline
%\newline
%\newline
%\newline
%\newline
%\[
%\text{PAGE 1 OF X FOR PROBLEM 4}
%\]
\end{problem}






\newpage
\begin{problem}{5}
Let $A$ be a 2$\times$2 matrix. Verify each of the following using the formula for the matrix-vector product.
\noindent
\newline
{\it Note:} Since matrices encode linear transformations, you should expect these to be true. In face, we can argue that they are true by interpreting the matrix as being the standard matrix of a certain linear transformation, and then just appealing to the fact that linear transformations preserve addition and scalar multiplication. However, in this problem, you should just work the through the computations directly. 
\noindent
\newline
\newline
a. $A(\vec{v_{1}}+\vec{v_{2}})=A\vec{v_{1}}+A\vec{v_{2}}$ for all $\vec{v_{1}},\vec{v_{2}} \in \mathbb{R}^2$.
\begin{solution}
Let $A=\begin{pmatrix} a&b\\c&d\end{pmatrix}$. Let $\vec{v_{1}},\vec{v_{2}} \in \mathbb{R}^2$ be arbitrary, and fix $x_{1},x_{2},y_{1},y_{2} \in \mathbb{R}$ with $\vec{v_{1}}=\begin{pmatrix}x_{1}\\x_{2}\end{pmatrix},\vec{v_{2}}=\begin{pmatrix}y_{1}\\y_{2}\end{pmatrix}$.
So we have
\begin{align*}
A(\vec{v_{1}}+\vec{v_{2}}) =& \begin{pmatrix} a&b\\c&d\end{pmatrix} \cdot \left(\begin{pmatrix}x_{1}\\x_{2}\end{pmatrix}+\begin{pmatrix}y_{1}\\y_{2}\end{pmatrix}\right) &\\
=&\begin{pmatrix} a&b\\c&d\end{pmatrix} \cdot \begin{pmatrix}x_{1}+y_{1}\\x_{2}+y_{2}\end{pmatrix} &\\
=&\begin{pmatrix} a\cdot (x_{1}+y_{1})+ b\cdot (x_{2}+y_{2})\\c\cdot (x_{1}+y_{1}) +d\cdot (x_{2}+y_{2})\end{pmatrix} &\text{(By definition of matrix-vector product)} \\
=& \begin{pmatrix} a\cdot x_{1}+ a\cdot y_{1}+ b\cdot x_{2}+b\cdot  y_{2}\\c\cdot x_{1}+ c\cdot y_{1}+d\cdot x_{2}+ d\cdot y_{2} \end{pmatrix} &\\
=& \begin{pmatrix} (a\cdot x_{1}+ b\cdot x_{2})+ (a\cdot y_{1}+b\cdot  y_{2})\\ (c\cdot x_{1}+ d\cdot x_{2})+(c\cdot y_{1}+ d\cdot y_{2}) \end{pmatrix} &\\
=& \begin{pmatrix} a\cdot x_{1}+ b\cdot x_{2}\\ c\cdot x_{1}+ d\cdot x_{2} \end{pmatrix} + \begin{pmatrix}  a\cdot y_{1}+b\cdot  y_{2}\\ c\cdot y_{1}+ d\cdot y_{2} \end{pmatrix} & \text{(By definition of matrix addition)}\\
=&\begin{pmatrix} a&b\\c&d\end{pmatrix} \cdot \begin{pmatrix}x_{1}\\x_{2}\end{pmatrix} +\begin{pmatrix} a&b\\c&d\end{pmatrix} \cdot \begin{pmatrix}y_{1}\\y_{2}\end{pmatrix} & \text{(By definition of matrix-vector product)}\\
=& A \vec{v_{1}}+ A \vec{v_{2}} &\text{(By definition of $A, \vec{v_{1}},\vec{v_{2}}$)}
\end{align*}
Therefore, $A(\vec{v_{1}}+\vec{v_{2}})=A\vec{v_{1}}+A\vec{v_{2}}$. Because $\vec{v_{1}},\vec{v_{2}}$ were arbitrary, the result follows.
\end{solution}
\noindent
\newline
\newline
\newline
\newline
\newline
\newline
\newline
\newline
\newline
\newline
\[
\text{PAGE 1 OF 2 FOR PROBLEM 5}
\]
\end{problem}






\newpage
\begin{problem}{6}
Consider the unique linear transformation $T:\mathbb{R}^2 \to \mathbb{R}^2$ such that
\begin{align*}
T\left(\begin{pmatrix}-1\\1\end{pmatrix}\right)=\begin{pmatrix}1\\4\end{pmatrix} && \text{ and } && T\left(\begin{pmatrix}-2\\3\end{pmatrix}\right)=\begin{pmatrix}2\\7\end{pmatrix} \text{.}
\end{align*}
What is $[T]$? Explain.
\noindent
\newline
\newline
\begin{solution}
Let $\vec{e_{1}}=\begin{pmatrix} 1\\0\end{pmatrix},\vec{e_{2}}=\begin{pmatrix} 0\\1\end{pmatrix}$
Notice that :
\begin{align*}
T\left(\begin{pmatrix}-1\\1\end{pmatrix}\right) =& T\left( \begin{pmatrix}-1\\0\end{pmatrix} + \begin{pmatrix}0\\1\end{pmatrix}\right) &\\
=&T\left( (-1)\cdot \begin{pmatrix}1\\0\end{pmatrix} \right)+ T\left(\begin{pmatrix}0\\1\end{pmatrix}\right) & \text{(By definition of linear transformation)}\\
=&T\left(\begin{pmatrix}0\\1\end{pmatrix}\right) - T\left(\begin{pmatrix}1\\0\end{pmatrix}\right) & \text{(By definition of linear Transformation)}\\
=&T(\vec{e_{2}})-T(\vec{e_{1}})=\begin{pmatrix}1\\4\end{pmatrix}& \text{(By definition of $\vec{e_{1}},\vec{e_{2}}, T\left(\begin{pmatrix}-1\\1\end{pmatrix}\right)$)} \text{,}
\end{align*}
and that
\begin{align*}
T\left(\begin{pmatrix}-2\\3\end{pmatrix}\right) =& T\left( \begin{pmatrix}-2\\0\end{pmatrix} + \begin{pmatrix}0\\3\end{pmatrix}\right) &\\
=&T\left( (-2)\cdot \begin{pmatrix}1\\0\end{pmatrix} \right)+ T\left(3\cdot \begin{pmatrix}0\\1\end{pmatrix}\right) & \text{(By definition of linear transformation)}\\
=&(-2)\cdot T\left(\begin{pmatrix}1\\0\end{pmatrix} \right)+ 3\cdot T\left(\begin{pmatrix}0\\1\end{pmatrix}\right) & \text{(By definition of linear transformation)}\\
=& 3T(\vec{e_{2}}) - 2 T(\vec{e_{1}})= \begin{pmatrix}2\\7\end{pmatrix} & \text{(By definition of $\vec{e_{1}},\vec{e_{2}},T\left(\begin{pmatrix}-2\\3\end{pmatrix}\right) $)}\text{.}
\end{align*}
So we have $T(\vec{e_{2}})-T(\vec{e_{1}})=\begin{pmatrix}1\\4\end{pmatrix}$, and $3T(\vec{e_{2}}) - 2 T(\vec{e_{1}})= \begin{pmatrix}2\\7\end{pmatrix}$. We solve for $\vec{e_{1}},\vec{e_{2}}$. We manipulate the first equation to get $T(\vec{e_{2}})=\begin{pmatrix}1\\4\end{pmatrix}+T(\vec{e_{1}})$ then substitute for $T(\vec{e_{2}})$ in the second equation:
\begin{align*}
3T(\vec{e_{2}}) - 2 T(\vec{e_{1}}) =& 3\cdot \left(\begin{pmatrix}1\\4\end{pmatrix}+T(\vec{e_{1}})\right)- 2 T(\vec{e_{1}}) &\\
=& \begin{pmatrix}3\\12\end{pmatrix} + 3T(\vec{e_{1}}) - 2 T(\vec{e_{1}}) &\\
=& \begin{pmatrix}3\\12\end{pmatrix} + T(\vec{e_{1}}) = \begin{pmatrix}2\\7\end{pmatrix}\text{.} \text{ Now we subtract $\begin{pmatrix}3\\12\end{pmatrix}$ from both sides to isolate $T(\vec{e_{1}})$:} &\\
\end{align*}
\[
\text{PAGE 1 OF 2 FOR PROBLEM 6}
\]
\end{solution}








\newpage
\noindent
and that
\begin{align*}
\begin{pmatrix}b\\d\end{pmatrix} =&\begin{pmatrix}\text{comp}_{\vec{e_{1}}}(C_{\theta}(\vec{e_{2}}))\\ \text{comp}_{\vec{e_{2}}}(C_{\theta}(\vec{e_{2}})) \end{pmatrix} &\\
=&\begin{pmatrix} \frac{\vec{e_{1}} \bullet C_{\theta}(\vec{e_{2}})}{\| \vec{e_{1}} \|} \\ \frac{\vec{e_{2}} \bullet C_{\theta}(\vec{e_{1}})}{\| \vec{e_{2}} \|} \end{pmatrix} &\text{(by definition of comp)} \\
=& \begin{pmatrix} \frac{\| \vec{e_{1}} \| \| C_{\theta}(\vec{e_{2}}) \| \cos (-\theta + \frac{\pi}{2})}{\| \vec{e_{1}} \|} \\ \frac{\| \vec{e_{2}} \| \| C_{\theta}(\vec{e_{2}}) \| \cos(-\theta)}{\| \vec{e_{2}} \|} \end{pmatrix} & \text{(by definition of dot product)}\\
=& \begin{pmatrix} \frac{1 \cdot 1 \cdot \cos (\frac{\pi}{2} -\theta)}{1} \\ \frac{1 \cdot 1 \cdot \cos (-\theta)}{1} \end{pmatrix} &\\
=& \begin{pmatrix} \cos (\frac{\pi}{2} -\theta)\\ \cos (-\theta) \end{pmatrix} & \\
=&  \begin{pmatrix} \cos (\frac{\pi}{2} -\theta)\\ \cos (\theta) \end{pmatrix} & \text{(by definition of even functions)}\\
=& \begin{pmatrix} \sin (\theta)\\ \cos (\theta) \end{pmatrix} & \text{(by properties of trig functions)}
\end{align*}
So $\begin{pmatrix}a\\c\end{pmatrix}=\begin{pmatrix} \cos (\theta)\\ -\sin (\theta) \end{pmatrix}=C_{\theta}(\vec{e_{1}}),\begin{pmatrix}b\\d\end{pmatrix}=\begin{pmatrix} \sin (\theta)\\ \cos (\theta) \end{pmatrix}=C_{\theta}(\vec{e_{2}})$. So by Definition 3.1.1, $[C_{\theta}]=\begin{pmatrix} \cos (\theta) & \sin(\theta) \\ -\sin(\theta) & \cos(\theta) \end{pmatrix}$. Notice that this is simply the negative of the counterclockwise rotation linear transformation as described in Chapter 3, which makes sense as multiplying a negative number by a linear transformation should cause it to act in the opposite direction, which in this case is a rotation the direction opposite of counterclockwise, which is clockwise.
\noindent
\newline
\newline
\newline
\newline
\newline
\newline
\newline
\newline
\newline
\newline
\newline
\newline
\newline
\newline
\newline
\newline
\newline
\newline
\[
\text{PAGE 2 OF 2 FOR PROBLEM 3}
\]








\newpage
\noindent
b. $A(r\cdot\vec{v})=r\cdot A\vec{v}$ for all $\vec{v} \in \mathbb{R}^2$ and all $r \in \mathbb{R}$.
\begin{solution}
Let $A=\begin{pmatrix} a&b\\c&d\end{pmatrix}$. Let $\vec{v} \in \mathbb{R}^2,r\in \mathbb{R}$ be arbitrary, and fix $x,y \in \mathbb{R}$ with $\vec{v}=\begin{pmatrix}x\\y\end{pmatrix}$.
So we have
\begin{align*}
A(r\cdot \vec{v}) =& \begin{pmatrix} a&b\\c&d\end{pmatrix} \cdot \left(r\cdot \begin{pmatrix}x\\y\end{pmatrix}\right) &\\
=&\begin{pmatrix} a&b\\c&d\end{pmatrix} \cdot \begin{pmatrix}rx\\ry\end{pmatrix} &\\
=&\begin{pmatrix} a\cdot (rx) + b\cdot (ry) \\c\cdot (rx) +d\cdot (ry)\end{pmatrix} &\text{(By definition of matrix-vector product)} \\
= &\begin{pmatrix} r\cdot (ax) + r\cdot (by) \\r\cdot (cx) +r\cdot (dy)\end{pmatrix} & \\
=& r\cdot \begin{pmatrix} ax +  by \\cx +dy\end{pmatrix} &\\
=& r\cdot \left(\begin{pmatrix} a&b\\c&d\end{pmatrix} \cdot \begin{pmatrix}x\\y\end{pmatrix}\right) & \text{(By definition of matrix-vector product)}\\
=& r\cdot A \vec{v} &\text{(By definition of $A, \vec{v}$)}
\end{align*}
Therefore, $A(r\cdot\vec{v})=r\cdot A\vec{v}$. Because $\vec{v}, r$ were arbitrary, the result follows.
\end{solution}
\noindent
\newline
\newline
\newline
\newline
\newline
\newline
\newline
\newline
\newline
\newline
\newline
\newline
\newline
\newline
\newline
\newline
\newline
\newline
\newline
\newline
\newline
\newline
\[
\text{PAGE 2 OF 2 FOR PROBLEM 5}
\]






\newpage
\noindent
\begin{align*}
T(\vec{e_{1}}) = &\begin{pmatrix}2\\7\end{pmatrix} - \begin{pmatrix}3\\12\end{pmatrix} \\
=& \begin{pmatrix}2-3\\7-12\end{pmatrix} = \begin{pmatrix} -1 \\-5\end{pmatrix}
\end{align*}
So $T(\vec{e_{1}}) = \begin{pmatrix} -1 \\-5\end{pmatrix}$. We substitute for $T(\vec{e_{1}})$ into our first equation:
\begin{align*}
T(\vec{e_{2}})=\begin{pmatrix}1\\4\end{pmatrix}+T(\vec{e_{1}})\\
=&\begin{pmatrix}1\\4\end{pmatrix} + \begin{pmatrix} -1 \\-5\end{pmatrix}\\
=& \begin{pmatrix} 1-1\\4-5\end{pmatrix} = \begin{pmatrix} 0\\ -1\end{pmatrix}
\end{align*}
So $T(\vec{e_{2}}) = \begin{pmatrix} 0 \\ -1\end{pmatrix}$. By Definition 3.1.1, we have that $[T] = \begin{pmatrix} -1 & 0\\ -5 & -1 \end{pmatrix}$. Now we use Proposition 3.1.4 to verify that we have the correct $[T]$:
\begin{align*}
T\left(\begin{pmatrix}-1\\1\end{pmatrix}\right)=[T]\begin{pmatrix}-1\\1\end{pmatrix}=&\begin{pmatrix} -1 & 0\\ -5 & -1 \end{pmatrix}\begin{pmatrix}-1\\1\end{pmatrix} &\\
=&\begin{pmatrix} -1\cdot -1 + 0\cdot 1\\ -5\cdot -1 + -1 \cdot 1 \end{pmatrix} \text{ (By definition of matrix-vector product)}& \\
=& \begin{pmatrix} 1+0\\ 5-1\end{pmatrix} =\begin{pmatrix} 1\\ 4 \end{pmatrix}, \text{ which agrees with the definition of $T\left(\begin{pmatrix}-1\\1\end{pmatrix}\right)$.} &
\end{align*}

\begin{align*}
T\left(\begin{pmatrix}-2\\3\end{pmatrix}\right) = [T]\begin{pmatrix}-2\\3\end{pmatrix} =& \begin{pmatrix} -1 & 0\\ -5 & -1 \end{pmatrix}\begin{pmatrix}-2\\3\end{pmatrix} &\\
=& \begin{pmatrix} -1\cdot -2 + 0 \cdot 3\\ -5\cdot -2 + -1\cdot 3 \end{pmatrix}\text{ (By definition of matrix-vector product)} & \\
=& \begin{pmatrix} 2 + 0 \\ 10 -3 \end{pmatrix} = \begin{pmatrix} 2 \\ 7 \end{pmatrix} \text{ which agrees with the definition of $T\left(\begin{pmatrix}-2\\3\end{pmatrix}\right)$.} &
\end{align*}
So our $[T]$ is indeed correct. Therfore, $[T] = \begin{pmatrix} -1 & 0\\ -5 & -1 \end{pmatrix}$.
\noindent
\newline
\newline
\newline
\newline
\newline
\newline
\newline
\newline
\newline
\newline
\[
\text{PAGE 2 OF 2 FOR PROBLEM 6}
\]
\end{problem}


\end{document}