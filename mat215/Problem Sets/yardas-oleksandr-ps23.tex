\documentclass[12pt]{article}
\usepackage{latexsym, amssymb, amsmath, amsfonts, amscd, amsthm, xcolor, pgfplots}
\usepackage{framed}
\usepackage[margin=1in]{geometry}
\linespread{1} %Change the line spacing only if instructed to do so.

\newenvironment{problem}[2][Problem]
{
	\begin{trivlist} 
		\item[\hskip \labelsep {\bfseries #1 #2:}]
	}
{
	\end{trivlist}
	}

\newenvironment{solution}[1][Solution]
{
	\begin{trivlist} 
		\item[\hskip \labelsep {\itshape #1:}]
	}
	{
	\end{trivlist}
}

\newenvironment{collaborators}[1][Collaborator(s)]
{
	\begin{trivlist} 
		\item[\hskip \labelsep {\bfseries #1:}]
	}
	{
	\end{trivlist}
}

%%%%%%%%%%%%%%%%%%%%%%%%%%%%%%%%%%%%%%%%%%%%%%%%%%
%%%%%%%%%%%%%%%%%%%%%%%%%%%%%%%%%%%%%%%%%%%%%%%%%%
%%%%%%%%%%%%%%%%%%%%%%%%%%%%%%%%%%%%%%%%%%%%%%%%%%
%
%
%    You need only modify code below this block.
%
%
%%%%%%%%%%%%%%%%%%%%%%%%%%%%%%%%%%%%%%%%%%%%%%%%%%
%%%%%%%%%%%%%%%%%%%%%%%%%%%%%%%%%%%%%%%%%%%%%%%%%%
%%%%%%%%%%%%%%%%%%%%%%%%%%%%%%%%%%%%%%%%%%%%%%%%%%
%
\title{Assignment: Problem Set 23} %Change this to the assignment you are submitting.
\author{Name: Oleksandr Yardas} %Change this to your name.
\date{Due Date: 05/09/2018 } %Change this to the due date for the assignment you are submitting.
\begin{document}
	\maketitle
	\thispagestyle{empty}
	
	\section*{List Your Collaborators:}%Enter your collaborators names below. Do not delete extra rows.
	
	\begin{itemize}
		\begin{framed}
			\item 
			Problem 1: None
			\\\\
		\end{framed}
		\begin{framed}
			\item 
			Problem 2: None
			\\\\
		\end{framed}
		\begin{framed}
			\item 
			Problem 3: None
			\\\\
		\end{framed}
		\begin{framed}
			\item 
			Problem 4: None
			\\\\
		\end{framed}
		\begin{framed}
			\item 
			Problem 5: None
			\\\\
		\end{framed}
		\begin{framed}
			\item 
			Problem 6: Not Applicable
			\\\\
		\end{framed}
	\end{itemize}
\newpage
%
%%%%%%%%%%%%%%%
%
% Your problem statements and solutions start here.
% Use the \newpage command between problems so that
% each of your problems begins on its own page.
%
%%%%%%%%%%%%%%%

%FORMATTING OPTIONS
%FOR BLANK SPACES: \underline{\hspace{2cm}}
%FOR SPACES IN align OR SIMILAR ENVIRONMENTS:  \hphantom{1000}
%FOR MATRICES: \begin{matrix} \end{matrix}, can add p, b, B, v, V, small as suffix to "matrix"
%SETS: \mathbb{R}^, :\mathbb{R}^ \to \mathbb{R}^
%Vectors: \vec{},
%SUBSCRIPTS: _{}
%FRACTIONS: \frac{}{}
%FANCY LETTERS: \mathcal{}

%Provide the problem statement.
\begin{problem}{1}
Calculate
\[
\begin{vmatrix} 3&0&-2&1\\4&0&2&0\\-5&2&-8&7\\3&0&3&-1\end{vmatrix}\text{.}
\]
\noindent
%a. [PART A STUFF]
\begin{solution}
We use the following facts to calculate $\begin{vmatrix} 3&0&-2&1\\4&0&2&0\\-5&2&-8&7\\3&0&3&-1\end{vmatrix}$:
\begin{align*}
\bullet& \text{If $B$ is obtained from $A$ by interchanging two rows, then $\text{det}(B)=-\text{det}(A)$}.\\
\bullet& \text{If $B$ is obtained from $A$ by multiplying a row by $c$, then $\text{det}(B)=c\cdot \text{det}(A)$.}\\
\bullet& \text{If $B$ is obtained from $A$ by row combination, then $\text{det}(B)=\text{det}(A)$.}
\end{align*}
So we get
\begin{align*}
\begin{vmatrix} 3&0&-2&1\\4&0&2&0\\-5&2&-8&7\\3&0&3&-1\end{vmatrix} =& \begin{vmatrix} 3&0&-2&1\\1&0&4&-1\\-5&2&-8&7\\3&0&3&-1 \end{vmatrix} & \begin{matrix} \text{} \\ -R_1+R_2\text{} \\ \text{} \\ \text{} \end{matrix} \\
%
=& \begin{vmatrix} 0&0&-14&4\\1&0&4&-1\\0&2&12&2\\0&0&-9&2 \end{vmatrix} & \begin{matrix} -3R_2+R_1\text{} \\ \text{} \\ 5R_2+R_3\text{} \\ -3R_2+R_4\text{} \end{matrix} \\
%
=& \begin{vmatrix} 0&0&0&\frac{8}{9}\\1&0&4&-1\\0&2&12&2\\0&0&-9&2 \end{vmatrix} & \begin{matrix} -\frac{14}{9}R_4 +R_1\text{} \\ \text{} \\ \text{} \\ \text{} \end{matrix} \\
%
=& \begin{vmatrix} 0&0&-9&2\\0&2&12&2\\1&0&4&-1\\0&0&0&\frac{8}{9} \end{vmatrix} & \begin{matrix} R_4 \leftrightarrow R_1 \\ R_3 \leftrightarrow R_2 \text{} \\ R_2 \leftrightarrow R_3 \text{} \\ R_1 \leftrightarrow R_4 \text{} \end{matrix} \\
%
=& (-1)\begin{vmatrix} 1&0&4&-1\\0&2&12&2\\0&0&-9&2\\0&0&0&\frac{8}{9} \end{vmatrix} & \begin{matrix} R_3 \leftrightarrow R_1 \\  \text{} \\ R_1 \leftrightarrow R_3 \text{} \\ \text{} \end{matrix} \\
=& (-1)(1\cdot 2 \cdot (-9) \cdot \frac{8}{9}) & \text{(By Proposition 5.3.10)}\\
=&(-1)(2\cdot (-8)) = 16&
\end{align*}
So $\begin{vmatrix} 3&0&-2&1\\4&0&2&0\\-5&2&-8&7\\3&0&3&-1\end{vmatrix} = 16$.
\end{solution}
%\vfill
%\centerline{PAGE 1 OF X FOR PROBLEM 1}\end{problem}
\end{problem}






\newpage
\begin{problem}{2}
Suppose that
\[
\begin{vmatrix} a&b&c\\d&e&f\\g&h&i \end{vmatrix} = 5\text{.}
\]
Find, with explanation, the value of
\[
\begin{vmatrix} -d&-e&-f\\2g+3a&2h+3b&2i+3c\\a&b&c \end{vmatrix}\text{.}
\]
\noindent
\newline
\newline
%a. [PART A STUFF]
\begin{solution}
We use the following facts to calculate $\begin{vmatrix} -d&-e&-f\\2g+3a&2h+3b&2i+3c\\a&b&c \end{vmatrix}$:
\begin{align*}
\bullet& \text{If $B$ is obtained from $A$ by interchanging two rows, then $\text{det}(B)=-\text{det}(A)$}.\\
\bullet& \text{If $B$ is obtained from $A$ by multiplying a row by $c$, then $\text{det}(B)=c\cdot \text{det}(A)$.}\\
\bullet& \text{If $B$ is obtained from $A$ by row combination, then $\text{det}(B)=\text{det}(A)$.}
\end{align*}
We perform elementary row operations on $\begin{vmatrix} -d&-e&-f\\2g+3a&2h+3b&2i+3c\\a&b&c \end{vmatrix}$:
\begin{align*}
\begin{vmatrix}  -d&-e&-f\\2g+3a&2h+3b&2i+3c\\a&b&c \end{vmatrix} =& (-1) \begin{vmatrix}2g+3a&2h+3b&2i+3c\\ -d&-e&-f\\a&b&c \end{vmatrix} & \begin{matrix} R_2 \leftrightarrow R_1 \text{} \\ R_1 \leftrightarrow R_2 \text{} \\  \text{} \end{matrix} \\
%
=& (-1)(-1)\begin{vmatrix} a&b&c\\ -d&-e&-f\\2g+3a&2h+3b&2i+3c \end{vmatrix} & \begin{matrix} R_3 \leftrightarrow R_1 \text{} \\ \text{} \\ R_1 \leftrightarrow R_3 \text{} \end{matrix} \\
%
=& (-1)(-1)(-1)\begin{vmatrix} a&b&c\\ d&e&f\\2g+3a&2h+3b&2i+3c \end{vmatrix} & \begin{matrix} \text{} \\ -R_2 \text{} \\ \text{} \end{matrix} \\
%
=& (-1)\begin{vmatrix} a&b&c\\ d&e&f\\2g&2h&2i \end{vmatrix} & \begin{matrix} \text{} \\ \text{} \\ -3R_1+R_3 \text{} \end{matrix} \\
%
=& (-1)(2)\begin{vmatrix} a&b&c\\ d&e&f\\g&h&i \end{vmatrix} & \begin{matrix} \text{} \\ \text{} \\ \frac{1}{2} R_3 \text{} \end{matrix} \\
=& -(2)(5) = -10 &
\end{align*}
So $\begin{vmatrix} -d&-e&-f\\2g+3a&2h+3b&2i+3c\\a&b&c \end{vmatrix}=-10$.
\end{solution}
%\vfill
%\centerline{PAGE 1 OF X FOR PROBLEM 2}
\end{problem}






\newpage
\begin{problem}{3}
Show that
\[
\begin{vmatrix} a&b&c\\a+x&b+x&c+x\\a+y&b+y&c+y \end{vmatrix} =0
\]
for all $a,b,c,x,y\in\mathbb{R}$.
\noindent
\newline
\newline
%a. [PART A STUFF]
\begin{solution}
Let $a,b,c,x,y\in\mathbb{R}$ be arbitrary. Consider $\begin{vmatrix} a&b&c\\a+x&b+x&c+x\\a+y&b+y&c+y \end{vmatrix}$. For a 3 $\times$ 3 matrix $\begin{pmatrix} a&b&c\\d&e&f\\g&h&i \end{pmatrix}$, we have that $\begin{vmatrix} a&b&c\\d&e&f\\g&h&i \end{vmatrix}=a(ei-hf)-b(di-gf)+c(dh-ge)$, so it follows that
\begin{align*}
&\begin{vmatrix} a&b&c\\a+x&b+x&c+x\\a+y&b+y&c+y \end{vmatrix}=\\
=&a((b+x)(c+y)-(b+y)(c+x))-b((a+x)(c+y)-(a+y)(c+x))\\+&c((a+x)(b+y)-(a+y)(b+x))\\
=& a(b+x)(c+y)-a(b+y)(c+x)-b(a+x)(c+y)+b(a+y)(c+x)+c(a+x)(b+y)-c(a+y)(b+x)\\
=& (ac+ay-ca-cy)(b+x) + (ba+by-ab-ay)(c+x)+(cb+cy-bc-by)(a+x)\\
=&(ay-cy)(b+x)+(by-ay)(c+x)+(cy-by)(a+x)\\
=&ayb+ayx-cyb-cyx+byc+byx-ayc-ayx+cya+cyx-bya-byx\\
=&ayb-bya+ayx-ayx+byc-cyb+cyx-cyx+byx-byx+cya-ayc\\
=&0+0+0+0+0+0=0
\end{align*}
So $\begin{vmatrix} a&b&c\\a+x&b+x&c+x\\a+y&b+y&c+y \end{vmatrix}=0$. Because $a,b,c,x,y\in\mathbb{R}$ were arbitrary, the result follows.
\end{solution}
%\vfill
%\centerline{PAGE 1 OF X FOR PROBLEM 3}
\end{problem}






\newpage
\begin{problem}{4}
Given $c\in\mathbb{R}$, consider the matrix
\[
A_c = \begin{pmatrix} 1&1&1\\1&9&c\\1&c&3\end{pmatrix} \text{.}
\]
\noindent
\newline
\newline
a. Use a cofactor expansion to compute det$(A_c)$.
\begin{solution}
We find the cofactors $C_{11},C_{12},C_{13}$. By Definition 5.3.13, we have that
\begin{align*}
C_{11}=&(-1)^{1+1}\cdot \begin{vmatrix}9&c\\c&3\end{vmatrix} = ((9)(3)-c^2 = 27-c^2\\
C_{12}=&(-1)^{1+2} \cdot \begin{vmatrix} 1&c\\1&3 \end{vmatrix} = (-1)(3-c) = c-3\\
C_{13}=&(-1)^{1+3} \cdot \begin{vmatrix} 1&9\\1&c \end{vmatrix} = c-9
\end{align*}
By Theorem 5.3.4, we have that
\begin{align*}
\begin{vmatrix} 1&1&1\\1&9&c\\1&c&3\end{vmatrix} =& 1\cdot C_{11} + 1\cdot C_{12} + 1\cdot C_{13}\\
=& 27-c^2 +c-3+c-9 = 15+2c-c^2
\end{align*}
So $\text{det}A_c = 15+2c-c^2$.
\end{solution}
%
\noindent
\newline
\newline
b. Find all values of $c$ such that $A_c$ is invertible. Explain.
\begin{solution}
By Corollary 5.3.11, $A_c$ is invertible if and only if $\text{det}(A_c) \neq 0$. So $A_c$ is invertible if and only if $15+2c-c^2\neq 0$. Using the quadratic equation, we find that $15+2c-c^2=0$ for $c=-3$ and $c=5$. So $A_c$ is invertible for all $c \in \mathbb{R} \setminus \{-3,5\}$.
\end{solution}
%\vfill
%\centerline{PAGE 1 OF X FOR PROBLEM 4}
\end{problem}






\newpage
\begin{problem}{5}
Find a basis for the eigenspace of the matrix
\[
\begin{pmatrix}1&4&1\\6&6&2\\-3&-4&-3 \end{pmatrix}
\]
corresponding to $\lambda = -2$.
\noindent
\newline
\newline
%a. [PART A STUFF]
\begin{solution}
Let $T:\mathbb{R}^3 \to \mathbb{R}^3$ be the linear transformation defined by letting $[T]=\begin{pmatrix}1&4&1\\6&6&2\\-3&-4&-3 \end{pmatrix}$. By Proposition 5.4.2, the eigenspace of $[T]$ corresponding to $\lambda=-2$ is the set $W=\{\vec{v} \in \mathbb{R}^3: T(\vec{v})= -2 \vec{v}\}$. So we want to find all eigenvectors $\vec{v}$ of $[T]$ corresponding to eigenvalue $-2$. Let $\vec{v} \in \mathbb{R}^3$ be arbitrary and nonzero, and fix $x,y,z \in \mathbb{R}$ with $\vec{v} = \begin{pmatrix} x\\y\\z \end{pmatrix}$. Suppose that $T(\vec{v})  =-2\vec{v}$. By Definition 5.3.1, we have that $\vec{v}$ is an eigenvector of $T$ corresponding to $\lambda =-2$. Notice that we have
\begin{align*}
\vec{0}_{\mathbb{R}^3} =& T(\vec{v}) - (-2)\vec{v}\\
=& [T]\vec{v} + 2 \vec{v}\\
=& ([T] +2 I_{\mathbb{R}^3}) \vec{v}
=&\left(\begin{pmatrix}1&4&1\\6&6&2\\-3&-4&-3 \end{pmatrix} + \begin{pmatrix}2&0&0\\0&2&0\\0&0&2 \end{pmatrix}\right) \begin{pmatrix} x\\y\\z \end{pmatrix}\\
=&\begin{pmatrix}3&4&1\\6&8&2\\-3&-4&-1 \end{pmatrix} \begin{pmatrix} x\\y\\z \end{pmatrix}\\
=& \begin{pmatrix}3x+4y+z\\6x+8y+2z\\-3x-4y-z \end{pmatrix}
\end{align*}
Notice that this is a linear system in the variables $x,y,z$, so to find $\vec{v}$ we just need to solve the system for $x,y,z$. We construct the augmented matrix and perform Gaussian Elimination:
\begin{align*}
\begin{pmatrix} 3&4&1&0\\6&8&2&0\\-3&-4&-1&0\end{pmatrix} \rightarrow & \begin{pmatrix} 3&4&1&0\\0&0&0&0\\0&0&0&0\end{pmatrix} &\begin{matrix} \text{} \\ -2R_1+R_2\text{} \\ R_1+R_3 \text{} \end{matrix}
\end{align*}
We obtain the solution $z=-3x-4y$. Introducing parameters $t=x,s=y$ we construct the solution set of the system: $\left\{ \begin{pmatrix}t\\0\\-3t\end{pmatrix} +  \begin{pmatrix} 0\\s\\-4s\end{pmatrix}: t,s \in \mathbb{R} \right\} = \text{Span}\left( \begin{pmatrix}1\\0\\-3\end{pmatrix}, \begin{pmatrix} 0\\1\\-4\end{pmatrix} \right)$. So $W = \left\{ \vec{v} \in \mathbb{R}^3: \vec{v} \in \text{Span}\left( \begin{pmatrix}1\\0\\-3\end{pmatrix}, \begin{pmatrix} 0\\1\\-4\end{pmatrix} \right) \right\} = \text{Span}\left( \begin{pmatrix}1\\0\\-3\end{pmatrix}, \begin{pmatrix} 0\\1\\-4\end{pmatrix} \right)$.
\vfill
\centerline{PAGE 1 OF 2 FOR PROBLEM 5}
%
\newpage
\noindent
We now show that $\left( \begin{pmatrix}1\\0\\-3\end{pmatrix}, \begin{pmatrix} 0\\1\\-4\end{pmatrix} \right)$ is linearly independent, and therefore a basis for $\text{Span}\left( \begin{pmatrix}1\\0\\-3\end{pmatrix}, \begin{pmatrix} 0\\1\\-4\end{pmatrix} \right)$. Consider the 3 $\times$ 2 matrix with $\begin{pmatrix}1\\0\\-3\end{pmatrix}$ and $\begin{pmatrix} 0\\1\\-4\end{pmatrix}$ as its columns. Performing Gaussian Elimination on this matrix, we get
\begin{align*}
\begin{pmatrix}1&0\\0&1\\-3&-4 \end{pmatrix} \rightarrow & \begin{pmatrix}1&0\\0&1\\0&0 \end{pmatrix} \begin{matrix} \text{} \\ \text{} \\ 3R_1+4R_2+R_3 \text{} \end{matrix}\\
\end{align*}
Notice that there is a leading entry in every column, so by Proposition 4.3.3 $\left( \begin{pmatrix}1\\0\\-3\end{pmatrix}, \begin{pmatrix} 0\\1\\-4\end{pmatrix} \right)$ is linearly independent. Therefore, $\left( \begin{pmatrix}1\\0\\-3\end{pmatrix}, \begin{pmatrix} 0\\1\\-4\end{pmatrix} \right)$ is a basis for the eigenspace of $[T]$ corresponding to $\lambda=-2$.
\end{solution}
\vfill
\centerline{PAGE 2 OF 2 FOR PROBLEM 5}
\end{problem}







\end{document}