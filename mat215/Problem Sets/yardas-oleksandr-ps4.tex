\documentclass[12pt]{article}
\usepackage{latexsym, amssymb, amsmath, amsfonts, amscd, amsthm}
\usepackage{framed}
\usepackage[margin=1in]{geometry}
\linespread{1} %Change the line spacing only if instructed to do so.

\newenvironment{problem}[2][Problem]
{
	\begin{trivlist} 
		\item[\hskip \labelsep {\bfseries #1 #2:}]
	}
{
	\end{trivlist}
	}

\newenvironment{solution}[1][Solution]
{
	\begin{trivlist} 
		\item[\hskip \labelsep {\itshape #1:}]
	}
	{
	\end{trivlist}
}

\newenvironment{collaborators}[1][Collaborator(s)]
{
	\begin{trivlist} 
		\item[\hskip \labelsep {\bfseries #1:}]
	}
	{
	\end{trivlist}
}

%%%%%%%%%%%%%%%%%%%%%%%%%%%%%%%%%%%%%%%%%%%%%%%%%%
%%%%%%%%%%%%%%%%%%%%%%%%%%%%%%%%%%%%%%%%%%%%%%%%%%
%%%%%%%%%%%%%%%%%%%%%%%%%%%%%%%%%%%%%%%%%%%%%%%%%%
%
%
%    You need only modify code below this block.
%
%
%%%%%%%%%%%%%%%%%%%%%%%%%%%%%%%%%%%%%%%%%%%%%%%%%%
%%%%%%%%%%%%%%%%%%%%%%%%%%%%%%%%%%%%%%%%%%%%%%%%%%
%%%%%%%%%%%%%%%%%%%%%%%%%%%%%%%%%%%%%%%%%%%%%%%%%%
%
\title{Assignment: Problem Set 4} %Change this to the assignment you are submitting.
\author{Name: Oleksandr Yardas} %Change this to your name.
\date{Due Date: 02/07/2018 } %Change this to the due date for the assignment you are submitting.
\begin{document}
	\maketitle
	\thispagestyle{empty}
	
	\section*{List Your Collaborators:}%Enter your collaborators names below. Do not delete extra rows.
	
	\begin{itemize}
		\begin{framed}
			\item 
			Problem 1: None
			\\\\
		\end{framed}
		\begin{framed}
			\item 
			Problem 2: None
			\\\\
		\end{framed}
		\begin{framed}
			\item 
			Problem 3: None
			\\\\
		\end{framed}
		\begin{framed}
			\item 
			Problem 4: None
			\\\\
		\end{framed}
		\begin{framed}
			\item 
			Problem 5: None
			\\\\
		\end{framed}
		\begin{framed}
			\item 
			Problem 6: None
			\\\\
		\end{framed}
	\end{itemize}
\newpage
%
%%%%%%%%%%%%%%%
%
% Your problem statements and solutions start here.
% Use the \newpage command between problems so that
% each of your problems begins on its own page.
%
%%%%%%%%%%%%%%%
%Provide the problem statement.
\begin{problem}{1}
	Describe the set $\{ x \in \mathbb{R} : |x| < 5 \} \cup \{ x \in \mathbb{R} : |x| \geq 3 \}$ more fundamentally without using set operations, and explain why your set is the same.
\end{problem}{

\begin{solution}
	By definition 1.5.9,
	\[
	A\cup B=\{x : x \in A \text{ or } x \in B\}
	\]
	So we have 
	\[
	\{x \in \mathbb{R} : |x| < 5 \} \cup \{ x \in \mathbb{R} : |x| \geq 3 \}=\{ x \in \mathbb{R} :  |x| < 5 \text{ or } |x| \geq 3 \}
	\]
	\begin{align*}
	=\{ x \in \mathbb{R} :  -5< x< 5 \text{ or } x \geq 3 \text{ or } x \leq 3 
	 \}\\
	 =\mathbb{R} %want this to be over next to the other "=" sign when I typeset it.
	\end{align*}
	The "or" quantifier allows the elements to be one of four things: greater than or equal to three, less than five, less than or equal to negative three, or greater than negative five. This means the elements can range from negative infinity to positive infinity, which is the same as the set $\mathbb{R}$. So we can say that $\{x \in \mathbb{R} : |x| < 5 \} \cup \{ x \in \mathbb{R} : |x| \geq 3 \}= \mathbb{R}$. 
\end{solution}

	
\newpage
\begin{problem}{2}
	Let $A= \{6n : n \in \mathbb{N}\} \cap \{10n : n \in \mathbb{N} \}$.
	\noindent
	a. Write down the smallest 3 elements of $A$, and briefly explain how you determined them.
	\begin{solution}
	By definition 1.5.10,
	\[
	A\cap B=\{x : x \in A \text{ and } x \in B\}
	\]
	So we have
	\[
	A= \{6n : n \in \mathbb{N}\} \cap \{10n : n \in \mathbb{N} \}
	\]
	\[
	= \{m \in \mathbb{N} : \text{There exists a } p \in \mathbb{N} \text { with } m=6p \} \cap \{m \in \mathbb{N} : \text{ There exists a } q \in \mathbb{N} \text { with } m=10q \}
	\]
	\[
	= \{m \in \mathbb{N} : \text{There exists } p,q \in \mathbb{N} \text { with } m=6p \text { and } m=10q \}
	\]
	This set is the set of all elements of $\mathbb{N}$ that have 6 and 10 as factors, that is, the set of all elements of $\mathbb{N}$ that have 10 as a factor which also have 6 as a factor. This would simply be the set of elements in $\mathbb{N}$ that are the elements in $\mathbb{N}$ multiplied by the lowest common multiple of 6 and 10, which is 30. More formally, this is
	\[
	A = \{n \in \mathbb{N} : \text { There exists a } m \in \mathbb{N} \text { with } m=30n \}
	A = \{30n : n \in \mathbb{N} \}
	\]
	So the three smallest elements of $A$ are 0, 30, and 60.
	\end{solution}
	\noindent
	b. Make a conjecture about how to describe $A$ parametrically (no need to prove this conjecture).
	\begin{solution}
	To describe this set parametrically, we simply have to define a parameter that give us multiples of 30 from $\mathbb{N}$. So we get
	\[
	A = \{30n : n \in \mathbb{N} \}
	\]
	\end{solution}
\end{problem}
		
\newpage
\begin{problem}{3}
	Given two sets $A$ and $B$, we define
	\[
	A \triangle B = \{x : x \text{ is an element of exactly one of } A \text{ or } B \},
	\]
	and we call this the {\it symmetric difference} of $A$ and $B$. For example, we have
	\[
	\{4,5,6,8\} \triangle \{5,6,7,8\} = \{4,7\}.
	\]
	
	\noindent a. Determine $ \{1,3,8,9\} \triangle \{2,3,4,7,8\}$.
	\begin{solution}
	Applying the definition of $A \triangle B$, we can see that the elements that are in exactly one of either set are 1, 2, 4, 7, and 9. So we get
	\[
	\{1,3,8,9\} \triangle \{2,3,4,7,8\} = \{1,2,4,7,9 \}
	\] 
	\end{solution}
	\noindent b. What are the smallest 9 elements of the set $\{2n:n \in \mathbb{N}\} \triangle \{3n : n \in \mathbb{N}\}$?
	\begin{solution}
	The elements of the set $\{2n:n \in \mathbb{N}\} \triangle \{3n : n \in \mathbb{N}\}$ are those elements which 2 and 3 do not have as a common multiple. So we get
	\[
	\{2n:n \in \mathbb{N}\} \triangle \{3n : n \in \mathbb{N}\} = \{0,2,4,6,8,10,12,14,16,18,20,22,...\}\triangle \{0,3,6,9,12,15,18,21,...\}
	\]
	\[
	= \{2,3,4,8,9,10,14,15,16,20,21,22,...\}
	\]
	It is easy to see that the smallest 9 elements of this set are 2,3,4,8,9,10,14,15, and 16.
	\end{solution}
	\noindent c. Make a conjecture about how to write $\{2n:n \in \mathbb{N}\} \triangle \{3n : n \in \mathbb{N}\}$ as the union of 3 pairwise disjoint sets (no need to prove this conjecture, but do write the 3 sets parametrically).
	\begin{solution}
	There seems to be a pattern to the set above. I can take the first 3 elements (2,3,4) and add 6 to each of them to get the next 3 elements (8,9,10), and I can add 6 to each of these to get the next three elements and so on. This means that I can describe this set as a union of the sets $\{2+6n : n \in \mathbb{N} \}$, $\{2+6n : n \in \mathbb{N} \}$, and $\{2+6n : n \in \mathbb{N} \}$:
	\[
	\{2+6n : n \in \mathbb{N} \} \cup \{3+6n : n \in \mathbb{N} \} \cup \{4+6n : n \in \mathbb{N} \}
	\]%, or more compactly:
	%\[
	%\{n \in \mathbb{N} : \text { There exists } m \in \mathbb{N} \text { with } n = 2+6m \text { or }  n = 3+6m \text { or }  n = 4+6m \}
	%\]
	\end{solution}

\end{problem}

\newpage
\begin{problem}{4}
	Define a funcion $f:\{1,2,3,...,12\} \to \mathbb{N}$ by letting $f(n)$ be the number of positive divisors of $n$. For example, the set of positive divisors of 6 is $\{1,2,3,6\}$, so $f(6)=4$.
	\noindent
	a. Write out $f$ formally as a set by listing all its elements.
	\begin{solution}
	\[
	f = \{(1,1),(2,2),(3,2),(4,3),(5,2),(6,4),(7,2),(8,4),(9,3),(10,4),(11,2),(12,6)\}
	\]
	
	%We will start by trying to find a pattern.
	%\begin{align*}
	%f(1)= 1 \\
	%f(2) = 2 \\
	%f(3) = 2 \\
	%f(4) = 3
	%f (5) = 2
	%f (60)
	%\end{align*}
	\end{solution}
	\noindent
	b. Write down the set range$(f)$ explicitly.
	\begin{solution}
	\[
	\text{range}(f)=\{1,2,3,4,6\}
	\]
	This seems to be the same as the positive divisors of $f(12)$.
	\end{solution}
\end{problem}

\newpage
\begin{problem}{5}
	Define a function $f:\mathbb{N} \times \mathbb{N} \to \mathbb{N}$ by letting $f((a,b)) = a^2 + b^2$. Does range$(f)=\mathbb{N}$? Explain you answer carefully.
\end{problem}
\begin{solution}
Let $f:\mathbb{N} \times \mathbb{N} \to \mathbb{N}$ be a function with $f((a,b)) = a^2 + b^2$. If range$(f) = \mathbb{N}$, then $f$ is surjective. For a function $f: A \to B$, The formal definition of surjective is below:
\[
f \text{ is surjective if, for all } b\in B, \text{ there exists } a \in A \text{ such that } f(a)=b \text{   (from Mileti)}
\]

So we need to prove (or provide a counter example) that $f:\mathbb{N} \times \mathbb{N} \to \mathbb{N}$ with $f((a,b)) = a^2 + b^2$ is surjective. We assume that $f$ is surjective. Let $n = 3$. So we can find an $x,y \in \mathbb{N}$ with $3 = x^2 + y^2$. Rearranging, we get $3 - x^2 = y^2$. We will do a case by case analysis.
\newline
\newline
\indent Consider the case in which $x^2 >3, x \in \mathbb{R}$. We have that $3-x^2<0$. We have a negative number equal to the square of $y$, so $y \in \mathbb{C}$. But $y \in \mathbb{N}$ by the definition of $f$. We have reached a contradiction, so if $f$ is surjective, we must have that $y^2 \leq 3, y \in \mathbb{N}$. 
\newline
\newline
\indent Now we look at the case in which $x^2 \leq 3, x \in \mathbb{N}$. 
So we must have that $x^2 =0$ or $1$. 
In the case that $x^2 = 1$, we have that $3-1=1=y^2$. 
But by theorem 1.4.8, there does not exist $q \in \mathbb{Q}$ with $q^2 = 2$, and and by definition, $\mathbb{N} \subseteq \mathbb{Q}$, so $\sqrt[]{2} \notin \mathbb{N}$. 
But $x \in \mathbb{N}$ by definition.
This case has led to a contradiction, so it must be the case that $x^2 < 2, x \in \mathbb{N}$. So $x^2=0$. We then have that $3-0=3=y^2$. It can be shown that there does not exist a there does not exist $q \in \mathbb{Q}$ with $q^2 = 3$, and by definition, $\mathbb{N} \subseteq \mathbb{Q}$, so $\sqrt[]{3} \notin \mathbb{N}$. But $x \in \mathbb{N}$ by definition. We again have a contradiction.

\indent We have exhausted all possible cases for $n=3$, so there is no $(x,y) \in \mathbb{N} \times \mathbb{N}$ with $3= x^2 + y^2$, so $f$ is not surjective. Therefore, range$(f) \neq \mathbb{N}$.
\newline
%\indent We now consider the case wher  So $y=\sqrt[]{2}$. By theorem 1.4.8, $\sqrt[]{2} \notin \mathbb{Q}$, and by definition, $\mathbb{N} \subseteq \mathbb{Q}$, so $\sqrt[]{2} \subsetneq \mathbb{R}$. But $y \in \mathbb{R}$ by definition, so $y^2 \neq 1$. Consider now the case in which
%Let us look at the set $C$ of the squares of the elements of $\mathbb{N}$, which contains all the elements of each ordered pair $(x,y) \in \mathbb{N} \times \mathbb{N}$.
%\[
%C= \{(n,n^2):n\in \mathbb{N}\}=\{0,1,4,9,16,25,.....\}
%\]


\end{solution}

\newpage
\begin{problem}{6}
	Consider the function $f:\mathbb{Q} \to \mathbb{Q}$ given by $f(a) = 5a-3$. We clearly have range$(f) \subseteq \mathbb{Q}$ by definition. Thus, to show $\mathbb{Q} =$ range$(f)$, it suffices to show $\mathbb{Q}\subseteq \text{ range}(f)$. To do this, we need to show how to take an arbitrary $b \in \mathbb{Q}$, and fill in the blank in $f(a)=b$ with an element of $\mathbb{Q}$. In this problem, we first do a few examples, and then handle a general $b$.
	\newline
	\newline 
	\noindent
	a. Fill in the blank in $f($\textunderscore \textunderscore \textunderscore \textunderscore $)=7$ with an element of $ \mathbb{Q} $.
	\begin{solution}
	Let $a \in \mathbb{Q}$.%$q=2$. So we get
	\begin{align*}
	f(a)=5*a -3 = 7\\
	10 = 5*a\\
	2=a
	\end{align*}
	So $a=2$.
	\end{solution}
	\noindent
	b. Fill in the blank in $f($\textunderscore \textunderscore \textunderscore \textunderscore $)=-53$ with an element of $\mathbb{Q}$.
	\begin{solution}
	Let $a \in \mathbb{Q}$.
	\begin{align*}
	f(a)=5*a -3 = -53\\
	-50 = 5*a\\
	-10=a
	\end{align*}
	So $a=-10$.

	%Let $q=\frac{-50}{5}$. So we get
	%\[
	%f(\frac{-50}{5})=5* \frac{-50}{5} -3 =-50 -3 =-53
	%\]
	\end{solution}
	\noindent
	c. Fill in the blank in $f($\textunderscore \textunderscore \textunderscore \textunderscore $) =1$ with an element of $\mathbb{Q}$.
	\begin{solution}
	Let $a \in \mathbb{Q}$.
	\begin{align*}
	f(a)=5*a -3 = 1\\
	4= 5*a\\
	\frac{4}{5}=a
	\end{align*}
	%Let $q=\frac{4}{5}$. So we get
	%\[
	%f(\frac{4}{5})=5* \frac{4}{5} -3 =-4 -3 =-1
	%\]
	So $a=\frac{4}{5}$.
	\end{solution}
	\noindent
	d. Let $b \in \mathbb{Q}$ be arbitrary. Fill in the blank in $f($\textunderscore \textunderscore \textunderscore \textunderscore $) =b$ with an element of $\mathbb{Q}$ (your answer will depend on $b$), and justify your choice of words.
	\begin{solution}
	Let $q \in \mathbb{Q}$ be arbitrary. So we have
	\begin{align*}
	f(q)=5*q -3 = b\\
	b+3= 5*q\\
	\frac{b+3}{5}=q
	\end{align*}
	So $q=\frac{b+3}{5}$.
	\newline
	\newline
	\noindent $\frac{b+3}{5} \in \mathbb{Q}$ because $b \in \mathbb{Q}$, and because $q,b$ were arbitrary, we conclude that $\mathbb{Q} \in \text{range}(f)$.
	\end{solution}

\end{problem}

\end{document}