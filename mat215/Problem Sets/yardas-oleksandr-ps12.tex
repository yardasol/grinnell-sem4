\documentclass[12pt]{article}
\usepackage{latexsym, amssymb, amsmath, amsfonts, amscd, amsthm, xcolor, pgfplots}
\usepackage{framed}
\usepackage[margin=1in]{geometry}
\linespread{1} %Change the line spacing only if instructed to do so.

\newenvironment{problem}[2][Problem]
{
	\begin{trivlist} 
		\item[\hskip \labelsep {\bfseries #1 #2:}]
	}
{
	\end{trivlist}
	}

\newenvironment{solution}[1][Solution]
{
	\begin{trivlist} 
		\item[\hskip \labelsep {\itshape #1:}]
	}
	{
	\end{trivlist}
}

\newenvironment{collaborators}[1][Collaborator(s)]
{
	\begin{trivlist} 
		\item[\hskip \labelsep {\bfseries #1:}]
	}
	{
	\end{trivlist}
}

%%%%%%%%%%%%%%%%%%%%%%%%%%%%%%%%%%%%%%%%%%%%%%%%%%
%%%%%%%%%%%%%%%%%%%%%%%%%%%%%%%%%%%%%%%%%%%%%%%%%%
%%%%%%%%%%%%%%%%%%%%%%%%%%%%%%%%%%%%%%%%%%%%%%%%%%
%
%
%    You need only modify code below this block.
%
%
%%%%%%%%%%%%%%%%%%%%%%%%%%%%%%%%%%%%%%%%%%%%%%%%%%
%%%%%%%%%%%%%%%%%%%%%%%%%%%%%%%%%%%%%%%%%%%%%%%%%%
%%%%%%%%%%%%%%%%%%%%%%%%%%%%%%%%%%%%%%%%%%%%%%%%%%
%
\title{Assignment: Problem Set 12} %Change this to the assignment you are submitting.
\author{Name: Oleksandr Yardas} %Change this to your name.
\date{Due Date: 03/12/2018 } %Change this to the due date for the assignment you are submitting.
\begin{document}
	\maketitle
	\thispagestyle{empty}
	
	\section*{List Your Collaborators:}%Enter your collaborators names below. Do not delete extra rows.
	
	\begin{itemize}
		\begin{framed}
			\item 
			Problem 1: None
			\\\\
		\end{framed}
		\begin{framed}
			\item 
			Problem 2: None
			\\\\
		\end{framed}
		\begin{framed}
			\item 
			Problem 3: None
			\\\\
		\end{framed}
		\begin{framed}
			\item 
			Problem 4: None
			\\\\
		\end{framed}
		\begin{framed}
			\item 
			Problem 5: None
			\\\\
		\end{framed}
		\begin{framed}
			\item 
			Problem 6: None
			\\\\
		\end{framed}
	\end{itemize}
\newpage
%
%%%%%%%%%%%%%%%
%
% Your problem statements and solutions start here.
% Use the \newpage command between problems so that
% each of your problems begins on its own page.
%
%%%%%%%%%%%%%%%

%FORMATTING OPTIONS
%FOR BLANK SPACES: \underline{\hspace{2cm}}
%FOR SPACES IN align OR SIMILAR ENVIRONMENTS:  \hphantom{1000}
%FOR MATRICES: \begin{matrix} \end{matrix}, can add p, b, B, v, V, small as suffix to "matrix"
%SETS: \mathbb{R}^, :\mathbb{R}^ \to \mathbb{R}^
%Vectors: \vec{},
%SUBSCRIPTS: _{}
%FRACTIONS: \frac{}{}
%PLUS/MINUS: \pm
%RIGHT/LEFT: 

%Provide the problem statement.
\begin{problem}{1}
Find the eigenvalues of the matrix
\[
\begin{pmatrix} 5&-1\\-7&3 \end{pmatrix} \text{.}
\]
\noindent
\newline
\newline
%a. [PART A STUFF]
\begin{solution}
We can find the eigenvalues $\lambda$ by solving the characteristic polynomial of $\begin{pmatrix} 5&-1\\-7&3 \end{pmatrix}$, that is, finding values of $\lambda$ such that the equation $(5-\lambda)(3-\lambda) - (-7)(-1) = 0$ is true. We assume that the characteristic polynomial is true. We then have
\begin{align*}
0=(5-\lambda)(3-\lambda) - (-7)(-1)=& (5-\lambda)(3-\lambda) - 7\\
=& 15-8 \lambda + {\lambda}^2 -7\\
=& {\lambda}^2 -8 \lambda + 8
\end{align*}
We complete the square, adding 8 to both sides:
\begin{align}
8 =&  \lambda ^2 -8 \lambda + 16\\
8 =& (\lambda -4)^2\\
\pm 2 \sqrt{2} =& \lambda - 4\\
4 \pm 2 \sqrt{2} =& \lambda
\end{align}
We have two eigenvalues, $\lambda_{1} = 4 + 2\sqrt{2}, \lambda_{2} = 4 - 2\sqrt{2}$.
\end{solution}
%\vfill
%\centerline{PAGE 1 OF X FOR PROBLEM 1}\end{problem}
\end{problem}






\newpage
\begin{problem}{2}
Find the eigenvalues of the matrix
\[
\begin{pmatrix}1&8\\2&1\end{pmatrix} \text{,}
\]
and then find (at least) one eigenvector for each eigenvalue.
\noindent
\newline
\newline
%a. [PART A STUFF]
\begin{solution}
We can find the eigenvalues $\lambda$ by solving the characteristic polynomial of $\begin{pmatrix} 1&8\\2&1 \end{pmatrix}$, that is, finding values of $\lambda$ such that the equation $(1-\lambda)^2 - (2)(8) = 0$ is true. We assume that the characteristic polynomial is true. We then have
\begin{align}
0=(1-\lambda)^2 - (2)(8)=& (1-\lambda)^2 - 16 \\
16=& (\lambda-1)^2\\
\pm 4 = \lambda-1\\
1 \pm 4 =\lambda
\end{align}
We have two eigenvalues, $\lambda_{1} = 1+4=5, \lambda_{2} = 1-4 =-3$. We now find eigenvectors of $\begin{pmatrix}1&8\\2&1\end{pmatrix}$ corresponding to eigenvalues $5$ and $-3$, that is, we find the value of the vectors $\vec{v_{1}}, \vec{v_{2}} \in \mathbb{R}^2$ that satisfy
\[
\left( \begin{pmatrix}1&8\\2&1\end{pmatrix} - 5I \right) \vec{v_{1}} = \vec{0}
 \hphantom{1000}
 \text{ and }
 \hphantom{1000}
\left( \begin{pmatrix}1&8\\2&1\end{pmatrix} +3I \right) \vec{v_{2}} = \vec{0}
\]
Letting, $\vec{v_{1}}=\begin{pmatrix} x_{1}\\y_{1}\end{pmatrix},\vec{v_{2}} = \begin{pmatrix} x_{2}\\y_{2}\end{pmatrix}$, we get,
\[
\begin{pmatrix}-4&8\\2&-4\end{pmatrix} \begin{pmatrix} x_{1}\\y_{1}\end{pmatrix} %= \vec{0}
 \hphantom{1000}
 \text{ and }
 \hphantom{1000}
\begin{pmatrix}4&8\\2&4\end{pmatrix}\begin{pmatrix} x_{2}\\y_{2}\end{pmatrix} %= \vec{0}
\]
which become
\[
\begin{pmatrix}-4x_{1}+8y_{1}\\2x_{1}-4y_{1}\end{pmatrix} %= \begin{pmatrix}0\\0\end{pmatrix}
 \hphantom{1000}
 \text{ and }
 \hphantom{1000}
\begin{pmatrix}4x_{2}+8y_{2}\\2x_{2}+4y_{2}\end{pmatrix}%= \begin{pmatrix}0\\0\end{pmatrix}
\]
Letting $x_{1}=2,y_{1}=1,x_{2}=-2,y_{2}=1$, we get
\[
\begin{pmatrix}-4\cdot 2+8\cdot 1\\2\cdot 2-4\cdot 1\end{pmatrix} =\begin{pmatrix}-8+8\\4-4\end{pmatrix} = \begin{pmatrix}0\\0\end{pmatrix}= \vec{0} %= \begin{pmatrix}0\\0\end{pmatrix}
 \hphantom{1000}
 \text{ and }
 \hphantom{1000}
\begin{pmatrix}4\cdot -2+8\cdot 1\\2\cdot -2+4\cdot 1\end{pmatrix} =\begin{pmatrix}-8+8\\-4+4\end{pmatrix}  = \begin{pmatrix}0\\0\end{pmatrix}=\vec{0} %= \begin{pmatrix}0\\0\end{pmatrix}
\]
So $\begin{pmatrix}2\\1\end{pmatrix}$ an eigenvector of $\begin{pmatrix}1&8\\2&1\end{pmatrix}$ corresponding to eigenvalue $-5$, and $\begin{pmatrix}-2\\1\end{pmatrix}$ an eigenvector of $\begin{pmatrix}1&8\\2&1\end{pmatrix}$ corresponding to eigenvalue $3$.
\end{solution}
%\vfill
%\centerline{PAGE 1 OF X FOR PROBLEM 2}
\end{problem}






\newpage
\begin{problem}{3}
Find the eigenvalues of the matrix
\[
\begin{pmatrix}2&-1\\1&4\end{pmatrix} \text{,}
\]
and then find (at least) one eigenvector for each eigenvalue.
\noindent
\newline
\newline
%a. [PART A STUFF]
\begin{solution}
We can find the eigenvalues $\lambda$ by solving the characteristic polynomial of $\begin{pmatrix}2&-1\\1&4\end{pmatrix}$, that is, finding values of $\lambda$ such that the equation $(2-\lambda)(4-\lambda) - (1)(-1) = 0$ is true. We assume that the characteristic polynomial is true. We then have
\begin{align*}
0=(2-\lambda)(4-\lambda) - (1)(-1)=& (2-\lambda)(4-\lambda) +1 \\
=& 8-6 \lambda + \lambda ^2 +1\\
=& \lambda ^2 -6 \lambda +9\\
=& (\lambda - 3)^2
\end{align*}
We have a repeated eigenvalue, $\lambda = 3$. We now find an eigenvector of $\begin{pmatrix}2&-1\\1&4\end{pmatrix}$ corresponding to eigenvalue $3$, that is, we find the value of the vectors $\vec{v} \in \mathbb{R}^2$ that satisfies
\[
\left( \begin{pmatrix}2&-1\\1&4\end{pmatrix} - 3I \right) \vec{v} = \vec{0}
\]
Letting, $\vec{v}=\begin{pmatrix} x\\y\end{pmatrix}$, we get,
\[
\begin{pmatrix}-1&-1\\1&1\end{pmatrix} \begin{pmatrix} x\\y\end{pmatrix} %= \vec{0}
\]
which becomes
\[
\begin{pmatrix}-x-y\\x+y\end{pmatrix} %= \begin{pmatrix}0\\0\end{pmatrix}
\]
Letting $x=1,y=-1$, we get
\[
\begin{pmatrix} -1-(-1)\\1+(-1)\end{pmatrix} = \begin{pmatrix} -1+1\\1-1\end{pmatrix} =\begin{pmatrix}0\\0\end{pmatrix} = \vec{0}
\]
So $\begin{pmatrix}1\\-1\end{pmatrix}$ an eigenvector of $\begin{pmatrix}2&-1\\1&4\end{pmatrix}$ corresponding to eigenvalue $3$.

\end{solution}
%\vfill
%\centerline{PAGE 1 OF X FOR PROBLEM 3}
\end{problem}






\newpage
\begin{problem}{4}
Find values for $c$ and $d$ such that the matrix
\[
\begin{pmatrix}3&1\\c&d\end{pmatrix}
\]
has both 4 and 7 as eigenvalues. You should show the derivation for how you arrived at your choice.
\noindent
\newline
\newline
%a. [PART A STUFF]
\begin{solution}
We want to find values of $c$ and $d$ such that the solutions of the characteristic polynomial are 4 and 7, that is, we want to find $c$ and $d$ such that the equation $0=(3-\lambda)(d-\lambda)-(c)(1)$ is true for $\lambda=4$ and $\lambda=7$. In other words, we need to find $c,d$ that solve the system of equations
\[
0=4-d-c \hphantom{1000} \text{ and } \hphantom{1000} 0=28-4d-c
\]
These equations are derived by plugging in 4 and 7 for $\lambda$, yielding
\[
0=(3-4)(d-4)-c \hphantom{1000} \text{ and } \hphantom{1000} 0=(3-7)(d-7)-c
\]
which become
\[
0=-1(d-4)-c \hphantom{1000} \text{ and } \hphantom{1000} 0=-4(d-7)-c \text{,}
\]
and finally
\[
0=4-d-c \hphantom{1000} \text{ and } \hphantom{1000} 0=28-4d-c \text{.}
\]
We solve for $d$:
\begin{align}
28-4d-c=&4-d-c\\
28-4d=&4-d\\
28-4=&4d-d\\
24=&3d\\
8=&d
\end{align}
So $d=8$. We now solve for $c$:
\begin{align}
0=28-32-c=&4-8-c=0\\
0=-4-c=&-4-c=0
\end{align}
We know that both the left and the right hand side have to be equal to zero, so it must be the case that $c=-4$. Thus, the values of $c$ and $d$ that give the matrix $\begin{pmatrix}3&1\\c&d\end{pmatrix}$ eigenvalues of 4 and 7 are $c=-4, d=8$.
\end{solution}
%\vfill
%\centerline{PAGE 1 OF X FOR PROBLEM 4}
\end{problem}






\newpage
\begin{problem}{5}
Consider the unique linear transfomation $T: \mathbb{R}^2 \to \mathbb{R}^2$ with
\[
[T] = \begin{pmatrix} 1&0\\6&-1\end{pmatrix} \text{.}
\]
Determine if $T$ is diagonalizable. If so, find an example of the basis $\alpha = (\vec{u_{1}}, \vec{u_{2}})$ of $\mathbb{R}^2$ such that $[T]_{\alpha}$ is a diagonal matrix, and determine $[T]_{\alpha}$ in this case.
\noindent
\newline
\newline
%a. [PART A STUFF]
\begin{solution}
By definition, $[T]$ is diagonalizable if there exists a basis $\alpha = (\vec{u_{1}},\vec{u_{2}})$ such that $[T]_{\alpha}$ is a diagonal matrix. It follows from Proposition 3.5.13, that $T$ is diagonalizable if and only if $\vec{u_{1}}$ and $\vec{u_{2}}$ are eigenvectors of $T$. So we need to find eigenvectors of $T$ $\vec{u_{1}}$ and $\vec{u_{2}}$ form a basis of $\mathbb{R}^2$. As we did before, we first find eigenvalues by solving the characteristic polynomial:
\begin{align*}
0=&(1-\lambda)(-1-\lambda)-(6)(0)\\
=&-(1-\lambda)(1+\lambda)\\
=&-(1-\lambda ^2) = \lambda ^2 -1
\end{align*}
We get $1=\lambda ^2$, which gives us two eigenvalues, $\lambda_{1}=1,\lambda_{2}=-1$. We now find the eigenvectors $\vec{v_{1}}=\begin{pmatrix} x_{1}\\y_{1}\end{pmatrix},\vec{v_{2}} = \begin{pmatrix} x_{2}\\y_{2}\end{pmatrix}$ corresponding to $\lambda_{1},\lambda_{2}$ respectively:
\begin{align}
\left( \begin{pmatrix} 1&0\\6&-1\end{pmatrix} -1I \right)\vec{v_{1}}=\vec{0} \hphantom{1000} \text{ and }& \hphantom{1000} \left( \begin{pmatrix} 1&0\\6&-1\end{pmatrix} +1I \right)\vec{v_{2}}=\vec{0} \\
\begin{pmatrix} 0&0\\6&-2 \end{pmatrix}\begin{pmatrix} x_{1}\\y_{1} \end{pmatrix}=\vec{0} \hphantom{1000} \text{ and }& \hphantom{1000} \begin{pmatrix} 2&0\\6&0\end{pmatrix}\begin{pmatrix} x_{2}\\y_{2}\end{pmatrix}=\vec{0}\\
\begin{pmatrix} 0\\6x_{1}-2y_{1} \end{pmatrix}=\vec{0} \hphantom{1000} \text{ and }& \hphantom{1000} \begin{pmatrix} 2x_{2}\\6x_{2}\end{pmatrix}=\vec{0}
\end{align}
It is easy to see that setting $\vec{v_{1}}=\begin{pmatrix}1\\3\end{pmatrix},\vec{v_{2}}=\begin{pmatrix}0\\1\end{pmatrix}$ satisfies the above equations. Note that $(1)(1)-(3)(0)=1\neq 0$, so by Theorem 2.3.10, Span$(\vec{v_{1}},\vec{v_{2}})=\mathbb{R}^2$. By definition of basis, $\beta=(\vec{v_{1}},\vec{v_{2}})$ is a basis of $\mathbb{R}^2$. By definition of $[T]_{\beta}$, the entries in the first row are the coordinates of $T(\vec{v_{1}})$ with respect to $\beta$ and the entries in the second row are the coordinates of $T(\vec{v_{2}})$ with respect to $\beta$. We find these as follows:
\begin{align*}
T(\vec{v_{1}})=\begin{pmatrix} 1&0\\6&-1\end{pmatrix}\begin{pmatrix}1\\3\end{pmatrix} = \begin{pmatrix}1\\3\end{pmatrix} = 1\vec{v_{1}} + 0\vec{v_{2}}\\
T(\vec{v_{2}})=\begin{pmatrix} 1&0\\6&-1\end{pmatrix}\begin{pmatrix}0\\1\end{pmatrix} = \begin{pmatrix}0\\-1\end{pmatrix} = 0\vec{v_{1}} - 1\vec{v_{2}}
\end{align*}
So $[T]_{\beta}=\begin{pmatrix}1&0\\0&-1\end{pmatrix}$, and it is clear to see that it is indeed diagonal.
\end{solution}
%\vfill
%\centerline{PAGE 1 OF X FOR PROBLEM 5}
\end{problem}






\newpage
\begin{problem}{6}
Consider the unique linear transfomation $T: \mathbb{R}^2 \to \mathbb{R}^2$ with
\[
[T] = \begin{pmatrix} 3&-1\\1&5\end{pmatrix} \text{.}
\]
Determine if $T$ is diagonalizable. If so, find an example of the basis $\alpha = (\vec{u_{1}}, \vec{u_{2}})$ of $\mathbb{R}^2$ such that $[T]_{\alpha}$ is a diagonal matrix, and determine $[T]_{\alpha}$ in this case.
\noindent
\newline
\newline
%a. [PART A STUFF]
\begin{solution}
By definition, $[T]$ is diagonalizable if there exists a basis $\alpha = (\vec{u_{1}},\vec{u_{2}})$ such that $[T]_{\alpha}$ is a diagonal matrix. It follows from Proposition 3.5.13, that $T$ is diagonalizable if and only if $\vec{u_{1}}$ and $\vec{u_{2}}$ are eigenvectors of $T$. So we need to find eigenvectors of $T$ $\vec{u_{1}}$ and $\vec{u_{2}}$ form a basis of $\mathbb{R}^2$. As we did before, we first find eigenvalues by solving the characteristic polynomial:
\begin{align*}
0=&(3-\lambda)(5-\lambda)-(1)(-1)\\
=&15-8\lambda +\lambda ^2 +1\\
=&\lambda ^2 -8\lambda +16 =(\lambda - 4)^2
\end{align*}
We get a repeated eigenvalue, $\lambda=4$, so $[T]$ only has one eigenvector, so there exists no basis $\alpha=(\vec{u_{1}},\vec{u_{2}})$ for which $\vec{u_{1}}$ and $\vec{u_{2}}$ are eigenvectors of $T$. We conclude that $[T]$ is not diagonalizable.
\end{solution}
%\vfill
%\centerline{PAGE 1 OF X FOR PROBLEM 6}
\end{problem}


\end{document}